\chapter{总结与展望}

本章总结全文的工作,并对展望未来的研究前景和进一步努力的方向。

\section{工作总结}
本文主要从参数计算与复杂性理论的角度研究NP难问题。
具体的研究对象有基于线性松弛下界的参数化顶点覆盖问题和参数化反馈顶点集问题。
顶点覆盖和反馈顶点集都是非常经典的组合优化领域的问题,也是图论中最重要的基础性问题之一。
他们在生物科学、电路设计、工程选址等现实领域都有着许多应用,因此这两个问题的固定参数算法设计具有重大理论意义和实际价值。

在基于线性松弛下界的参数化顶点覆盖问题(VCAL)上,
我们参考了之前Iwata等的研究成果\upcite{iwata2014linear},
首先将线性松弛下的最小顶点覆盖问题(LPVC)转化成最大网络流问题。
之后,仔细讨论了该网络流模型跟原问题之间的联系,并设计了利用网络流模型来对原问题实例进行收缩的核心化算法。
注意借助网络流模型,我们成功把这一阶段的时间复杂度控制在$O(\abs{V} + \abs{E})$内。
最后,在核心化之后的问题实例上,我们采用分支搜索技术进行求解。
通过对当前图结构的仔细讨论,我们设计了3条搜索分支规则,使得搜索树的规模被限定在$O(2.619^\mu)$内。
另外,在这个阶段中我们仔细确保了所有规则的执行时间以及执行规则后维护新的最大流的时间均是线性的。
结合以上所有的工作,我们成功获得了一个时间复杂度为$O(2.618^\mu(\abs{V} + \abs{E}))$的线性固定参数算法,
对比之前该问题上最好的线性固定参数算法$O(4^\mu(\abs{V} + \abs{E}))$,有十分明显的提升。


在参数化反馈顶点集问题上,我们在CaoYixin等\upcite{cao2010feedback}等的算法基础上进行了进一步研究。
主要思路是,先使用迭代压缩技术将FVS问题转化成该问题的一个变种,Disjoint-FVS问题,
这是当前求解参数化反馈顶点集问题的一个标准框架。
之后在Disjoint-FVS问题的求解上,我们创新性地同时使用两个参数对搜索树的规模进行约束,得到了一个与其他相关文献都不一样的结论。
最终,获得了一个时间复杂度为$\mathcal{O}^*(3.598^k)$的FPT算法,大大地改进了原来CaoYixin等$\mathcal{O}^*(3.83^k)$的结论。
如果横向对比其他同样改进原来CaoYixin等算法的其他研究成果,
当前最好的是Kociumaka等\upcite{2013arXiv1306.3566K}的成果,他们的时间复杂度是$\mathcal{O}^*(3.592^k)$,与我们优化的思路不一致但是结果十分接近。


\section{可改进点及未来研究展望}
关于本文所研究的两个问题,依然有很多有待改进的地方,有待于我们的进一步研究,主要有以下几个方面:

\textbf{(1)参数化顶点覆盖问题更低的时间复杂度}

在基于线性松弛下界的参数化顶点覆盖问题(VCAL)上,如果只考虑$f(\mu)$函数不考虑多项式部分,
当前已经有研究结果可以做到低于$2.619^\mu$了\upcite{lokshtanov2014faster},对于这样的算法能不能也对其多项式部分进行改进,将复杂度降到线性,以提高算法的可用性?
基于以上考虑,后续研究中可以尝试设计复杂度为$O(2.3146^\mu(\abs{V} + \abs{E}))$或者更优秀的算法。

\textbf{(2)带权图上的参数化顶点覆盖问题}

本文中,基于线性松弛下界的参数化顶点覆盖问题(VCAL)的讨论都是集中在不带权的普通图上,
我们的算法并不能直接移植到带权图上,这个是我们研究工作里面比较遗憾的缺陷。
后续工作中可以对带权图上该问题进行进一步研究。

\textbf{(3)参数化反馈顶点集问题的进一步研究思路}

在该问题的确定性FPT算法地设计中,我们和Kociumaka等分别从不同的思路对之前的算法进行了研究,分别获得了十分接近的新结论。
而这两个思路是否能结合起来,相辅相成获得新的理论下界?这也是后续工作中一个重要的方向。
另外,在随机FPT算法里该问题的复杂度到达了$\mathcal{O}^*(3^k)$,我们相信肯定也有确定性的FPT算法能到达这个复杂度,但还有待后续的研究。