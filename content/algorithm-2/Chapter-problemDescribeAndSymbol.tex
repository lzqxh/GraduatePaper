\chapter{基于Lov{\'a}sz-Plummer下界的参数化顶点覆盖问题}

\section{相关术语及问题的定义}

\subsection{基于Lov{\'a}sz-Plummer下界的参数化顶点覆盖问题定义}
在上一大章中,我们优化了基于线性松弛下界的参数化顶点覆盖问题,提出了一个更快的线性固定参数算法。
然而,实际上关于顶点覆盖值还存在着其他下界,比如说图的最大匹配数(简称,$MM(G)$),Lov{\'a}sz-Plummer下界(等于$2vc^*(G) - MM(G)$)。
关于这个两个下界也引申出类似的参数化顶点覆盖问题,如:\\

\begin{tabular}{| p{0.9\headwidth} |}
  \hline
  基于最大匹配下界的参数化顶点覆盖问题\\(Above-Guarantee Vertex Cover,简称AGVC) \\
  \textbf{输入:} 图$G(V, E)$及整数k \\
  \textbf{参数:} $\tilde{\mu} = k - MM(G)$\\
  \textbf{问题:} 图G中是否存在一个顶点覆盖集合,其大小不超过k\\
  \hline
\end{tabular} \vspace{0.5cm} \\


\begin{tabular}{| p{0.9\headwidth} |}
  \hline
  基于Lov{\'a}sz-Plummer下界的参数化顶点覆盖问题\\(Vertex Cover Above Lov$\acute{a}$sz-Plummer,简称VCAL-P) \\
  \textbf{输入:} 图$G(V, E)$及整数k \\
  \textbf{参数:} $\hat{\mu} = k - (2vc^*(G) - MM(G))$\\
  \textbf{问题:} 图G中是否存在一个顶点覆盖集合,其大小不超过k\\
  \hline
\end{tabular} \vspace{0.5cm} \\

在经典匹配理论中,Lov{\'a}sz和Plummer\upcite{lovasz2009matching} 已经证明了
\begin{itemize}
  \item[(1)]$2vc^*(G) - MM(G) \le vc(G)$;
  \item[(2)]$2vc^*(G) - MM(G) \ge vc^*(G) \ge MM(G)$。
\end{itemize}
这里利用上诉结论,对比VCAL,AGVC,VCAL-P三个问题的参数显然有,\[\tilde\mu(G, k) \ge \mu(G,k) \ge \hat\mu(G,k)\]
因此对于同一个图,上述三者里VCAL-P问题的固定参数算法其参数是最小的,即其参数是更加紧致(stricter)的。
另外据我们所知,VCAL-P是所有参数化顶点覆盖问题里,参数最紧致的。


在文献\cite{garg2015raising}中,作者最早定义了VCAL-P问题,证明了其是固定参数可解并给出了一个$\mathcal{O}^*(3^{\hat\mu})$的FPT算法。
我们将在前人的研究基础上继续优化该问题,并成功获得时间复杂度为\textcolor[rgb]{1.00,0.00,0.00}{补充}的固定参数算法,较之前的最好结果有了较大的改进。

\subsection{有关图匹配的一些术语及性质}
在本章之中,我们将继续使用上一章定义的关于图的术语,同时本节将再给出一些新的术语和一些已经被前人证明的性质。

\begin{definition}[匹配集]
对于图$G(V, E)$,如果其一个边集$E' \subseteq E$满足对其中任意两条边$e_1, e_2 \in E'$有$e_1 \cap e_2 = \emptyset$,
则我们将边集$E'$定义为图$G$的一个匹配。如果$E'$是所有匹配中最大的那个,则称为图G的最大匹配。
所有的顶点都在$E'$中,则称该匹配为图G的完美匹配。
\end{definition}

\begin{definition}[因子临界图(factor-critical graph)]
对于图$G(V, E)$,如果对于任意一个顶点$v \in V$,其导出子图$G \setminus \{v\}$都有完美匹配,则图G是因子临界图。
\end{definition}

求解一般图的最大匹配是图论中非常经典的问题,早在1980年Micali等已经给出了一个多项式算法\upcite{micali1980v}。 
同时,除了图的最大匹配,我们算法中还会使用到一个一般图匹配问题上的经典分解,如下:

\begin{definition}[Gallai-Edmonds分解]
对于图$G(V, E)$,其Gallai-Edmonds分解是它顶点集合的一个划分,$V = O \uplus I \uplus P$,其中3个子集分别定义如下:
\begin{itemize}
  \item $O = \{v \in V\;|\;\text{存在图G的最大匹配不包含顶点$v$} \}$
  \item $I = N(O)$
  \item $P = V \setminus (O \cup I)$
\end{itemize}
\end{definition}

接下来,罗列出我们算法中会使用到的关于Gallai-Edmonds分解的性质\upcite{lovasz2009matching,edmonds1965paths,gallai1963kritische,gallai1964maximale}。

\begin{lemma}
对于图$G(V, E)$,其Gallai-Edmonds分解是唯一的,并且能在多项式时间内获得。
\end{lemma}

\begin{lemma}
假设$V = O \uplus I \uplus P$是图$G(V, E)$的Gallai-Edmonds分解,则下列性质均成立
\begin{itemize}
  \item 图G的导出子图$G[O]$中,每一个连通块都是因子临界图;
  \item 图G的一个匹配M是其最大匹配当且仅当
        \begin{itemize}
          \item[a.]对于图$G$的导出子图$G[O]$中每个连通块$H$,边集$M \cap E(H)$构成H的一个近似完美匹配;
          \item[b.]对于每个顶点$i \in I$,存在另外一个顶点$o \in O$使得$\{i, o\} \in M$;
          \item[c.]对于图$G$的导出子图$G[P]$,边集$M \cap E(G[P])$构成其完美匹配。
        \end{itemize}
\end{itemize}
\end{lemma}

\begin{lemma}[Gallai-Edmonds分解的稳定性] \label{stability lemma}
用$O(H), I(H), P(H)$ 表示图H的Gallai-Edmonds分解的三个子集。则对于任意一个顶点$v \in V(G)$,有
\begin{itemize}
\item 如果$v \in O(G)$, 那么\[O(G - v) \subseteq (O(G) \setminus \{ v \}), I(G - v) \subseteq I(G),P(G - v) \supseteq P(G)\]
\item 如果$v \in I(G)$, 那么\[O(G - v) = O(G), I(G - v) = (I(G) \setminus \{ v \}),P(G - v) = P(G).\]
\item 如果$v \in P(G)$, 那么\[O(G - v) \supseteq O(G), I(G - v) \supseteq I(G),P(G - v) \subseteq (P(G) \setminus \{ v \}).\]
\end{itemize}
\end{lemma}

\subsection{有关可满足性问题(SAT)的一些术语的定义}
可满足性问题(SAT问题)是指给定一个布尔表达式,问是否存在一组变量的赋值使得表达式是可以满足的。
本文中讨论的SAT问题,均是其中比较特殊的,如果没有

