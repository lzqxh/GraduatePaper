\chapter{反馈顶点集问题的固定参数算法}
反馈顶点集问题(Feedback Vertex Set, 缩写为FVS),即对于参数$k$,询问给定图中是否存在大小满足$k$的顶点集合满足:
对于图中任意一条环都与这个顶点集有交集。
与顶点覆盖问题类似,该问题也是最早被提出来的$21$个NP完全问题之一\upcite{karp1972reducibility}。

本章里对于该问题的经典参数化版本进行了研究,并且提出了一个时间复杂度为$\mathcal{O}^*(3.598^k)$的固定参数算法。
该算法与Kociumaka的算法\upcite{kociumaka2014faster}一样,都是在之前Cao和Chen等的算法\upcite{cao2010feedback}的算法的基础上进行优化。
其中,我们与这两个文献最主要的区别是我们引入了双参数来评估Disjoint-FVS问题实例\footnote{Disjoint-FVS问题是在原FVS问题上应用迭代压缩技术后出现的一个中间问题,后文中会给出详细定义。}。

\section{相关术语及问题的定义}
本节中,我们定义了算法中会使用的图论相关的术语\footnote{注意本章里我们会继续沿用上一大章中关于图的符号和术语定义,对于上一章已经定义过的术语,我们不会在本节中重复定义,而是直接使用。}
并且引进了部分前人已经证明的结论。
此外对于算法所解决的问题也给予了更加数学的全面的定义。


\subsection{有关图的一些术语的定义}
首先给出无向图上一些常见名词的定义,如环,树和森林等。
\begin{definition}[环]
对于无向图$G(V,E)$,假设存在一个顶点序列$v_1v_2...v_n$,使得$\{v_0, v_n\} \in E$并且对于任意$0 \le i < n$有$\{v_i, v_{i+1}\} \in E$。
则该序列为图$G$上的一个\textbf{环}。如果该序列中没有重复顶点,则称之为图$G$的一个\textbf{简单环}。
\end{definition}

\begin{definition}[树(Tree)和森林(Forest)]
对于无向图$G(V,E)$,如果其中不存在任何环,则我们称图$G$为一个\textbf{森林(forest)}。
同时如果图$G$是连通的,则其也被称为\textbf{树(tree)}。
\end{definition}

\begin{definition}[反馈顶点集]
对于图$G(V, E)$, 若集合$V' \subseteq V$使得$G \setminus V'$中不存在任何环(即,图$G \setminus V'$是森林),
则称$V'$是图$G$的一个\textbf{反馈顶点集}(Feedback Vertex Set, 简称$FVS$)。
\end{definition}

为了方面对于算法的描述,我们继续引进一些图的术语。对于图$G$,$cc(G)$表示图$G$上连通块的个数。

\subsection{参数化反馈顶点集问题}
对于一般图来说,回答是否存在大小不超过$k$的反馈顶点集,是非常经典的组合优化问题。
在该问题上,固定参数算法是常用来求解答案的方法。
具体参数化反馈顶点集问题表述如下:\\

\begin{tabular}{| p{0.9\headwidth} |}
  \hline
  参数化反馈顶点集问题(Feedback Vertex Set,简称FVS) \\
  \textbf{输入:} 图$G(V, E)$及整数$k$ \\
  \textbf{参数:} $k$\\
  \textbf{问题:} 图$G$中是否存在一个反馈顶点集合,其大小不超过$k$\\
  \hline
\end{tabular} \vspace{0.5cm} \\

在参数计算领域,参数化FVS也是其中一个非常重要的问题,
一系列的研究一直在不停的改进其下限,如文献\cite{bodlaender1994disjoint,downey1992fixed,downey2012parameterized,raman2006faster,kanj2004parameterized,dehne20072o,guo2006compression,chen2008improved,cao2010feedback,cygan2011solving,kociumaka2014faster} 等。

在文献\cite{cygan2011solving}中作者Cygan等利用其发明的Cut\&Count技术,设计了时间复杂度为$\mathcal{O}^*(3^k)$的固定参数算法来求解FVC问题。
该算法是一个蒙特卡罗算法(Monte Carlo algorithm),如果不存在大小不超过$k$的FVS集合,那么算法一定会返回“NO”,否则有$1/2$概率返回“YES”。
据我们所知,该算法是迄今为止FVS问题上复杂度最低的固定参数算法。
而如果不希望使用随机性算法的话,迄今最快的算法是Kociumaka等\upcite{kociumaka2014faster}在2014年提出的基于迭代压缩技术的算法,
他们可以在$\mathcal{O}^*(3.592^k)$的时间复杂度内确切的给出问题的答案。

本章中,我们同样使用迭代压缩技术,最后获得一个$\mathcal{O}^*(3.598^k)$时间复杂度的固定参数算法,
对比Kociumaka的算法,时间复杂度上我们十分接近,然而Kociumaka的算法一共使用了$12$条分支规则,而我们的算法只有$1$条更加简洁明了。

关于迭代压缩技术(Iterative Compression Technology),是指将原问题拆分成一系列规模逐渐增长的子问题,并依次对这些子问题进行求解。
这样可以保证在求解每个子问题的时候,我们便已经知道上一个与其规模十分接近的子问题的解,利用上个问题的解来降低问题求解当前问题的难度。
很多固定参数算法的设计都使用了该技术。

在本文中我们在求解原反馈顶点集问题分解出来的子问题时,利用了不相交反馈顶点集问题(Disjoint Feedback Vertex Set,简称Disjoint-FVS问题),
这里我们给出它详细的定义,具体使用方法我们留在下一节中。
该问题是在原一般图上FVS问题增加额外的约束获得的,要求图$G$可以分解成两个森林,并且所求反馈顶点集必须只取其中一个森林中的顶点,
详细定义如下:\\

\begin{tabular}{ | p{0.06\headwidth} p{0.80\headwidth} | }
  \hline
  \multicolumn{2}{|l|}{ \textbf{不相交反馈顶点集问题} }\\
  \multicolumn{2}{|l|}{ (Disjoint Feedback Vertex Set,简称Disjoint-FVS)}\\
  \textbf{输入:} & 图$G(V, E)$,整数$k$ 以及顶点集合$V$的划分$V = U \uplus D$满足$G[U]$和$G[D]$都是森林\\
  \textbf{问题:} & 图$G$中是否存在一个反馈顶点集合$X \subseteq D$,其大小不超过$k$\\
  \hline
\end{tabular} \vspace{0.5cm}

