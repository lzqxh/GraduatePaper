\chapter{反馈顶点集问题的固定参数算法}
反馈顶点集问题(Feedback Vertex Set, 缩写为FVS),即对于参数k,询问给定图中是否存在大小满足k的顶点集合满足:
对于图中任意一条环都与这个顶点集有交集。
与顶点覆盖问题类似,该问题也是最早被提出来的21个NP-complete问题之一\cite{karp1972reducibility}。

本章提出了一个时间复杂度为$\mathcal{O}^*(2.554^k)$的固定参数算法。
\section{相关术语及问题的定义}

本章里我们会继续沿用上一大章中关于图的符号和术语。同时,在本节中我们会继续扩充一些会使用到的术语,
并且给出相关问题更加具体的数学的定义。
\subsection{有关图的一些术语的定义}
首先给出无向图上一些常见名词的定义,如环,树和森林等。
\begin{definition}[环]
对于无向图$G(V,E)$,假设存在一个顶点序列$v_1v_2...v_n$,使得$\{v_0, v_n\} \in E$并且对于任意$0 \le i < n$有$\{v_i, v_{i+1}\} \in E$。
则该序列为图G上的一个\textbf{环}。如果该序列中没有重复顶点,则称之为图G的一个\textbf{简单环}。
\end{definition}

\begin{definition}[树(tree)和森林(forest)]
对于无向图$G(V,E)$,如果其中不存在任何环,则我们称图G为一个\textbf{森林(forest)}。
同时如果图G是连通的,则其也被称为\textbf{树(tree)}。
\end{definition}

\begin{definition}[反馈顶点集]
对于图$G(V, E)$, 若集合$V' \subseteq V$使得$G \setminus V'$中不存在任何环(即,图$G \setminus V'$是森林),
则称$V'$是图G的一个\textbf{反馈顶点集}(Feedback Vertex Set, 简称$FVS$)。
\end{definition}

为了方面对于算法的描述,我们继续引进一些图的术语。对于图G,$cc(G)$表示图G上连通块的个数。

\subsection{参数化反馈顶点集问题}
对于一般图来说,回答是否存在大小不超过k的反馈顶点集,是非常经典组合优化问题。
在该问题上,固定参数算法是常用来求解答案的方法。
具体参数化反馈顶点集问题表述如下:\\

\begin{tabular}{| p{0.9\headwidth} |}
  \hline
  参数化反馈顶点集问题(Feedback Vertex Set,简称FVS) \\
  \textbf{输入:} 图$G(V, E)$及整数k \\
  \textbf{参数:} $k$\\
  \textbf{问题:} 图G中是否存在一个反馈顶点集合,其大小不超过k\\
  \hline
\end{tabular} \vspace{0.5cm} \\

在参数计算领域,参数化FVS也是其中一个非常重要的问题,
一系列的研究一直在不停的改进其下限,如文献\cite{bodlaender1994disjoint,downey1992fixed,downey2012parameterized,raman2006faster,kanj2004parameterized,dehne20072o,guo2006compression,chen2008improved,cao2010feedback,cygan2011solving,kociumaka2014faster} 等。

在文献\cite{cygan2011solving}中作者Cygan等利用其发明的Cut\&Count技术,设计了时间复杂度为$\mathcal{O}^*(3^k)$的固定参数算法来求解FVC问题。
该算法是一个蒙地卡罗算法(Monte Carlo algorithm),如果不存在大小不超过k的FVS集合,那么算法一定会返回“NO”,否则有1/2概率返回“YES”。
据我们所知,该算法是迄今为止FVS问题上复杂度最低的固定参数算法。
而如果不希望使用随机性算法的话,迄今最快的算法是Kociumaka等\upcite{kociumaka2014faster}在2014年提出的基于迭代压缩技术的算法,
他们可以在$\mathcal{O}^*(3.619^k)$的时间复杂度内确切的给出问题的答案。

本章中,我们同样使用迭代压缩技术,最后获得一个\textcolor[rgb]{1.00,0.00,0.00}{$\mathcal{O}^*(2.554^k)$}时间复杂度的固定参数算法。

