\section{Disjoint-FVS问题算法} \label{disjoint-fvs-algorithm}
求解Disjoint-FVS问题的方法,一般都是分支搜索算法,基本思路是取出顶点集合D中的某个顶点v,通过假设v是否存在于反馈顶点集中,我们可以获得实例的两个分支。
本章中,我们将给出一个Disjoint-FVS问题的算法,以证明定理\ref{Disjoint-FVSTime}的正确性。

\subsection{Disjoint-FVS问题实例的评估}
我们的算法与之前Cao\upcite{cao2010feedback}和Kociumaka\upcite{kociumaka2014faster}的算法相似,
与这两个算法相比,我们创新性地使用一个二元组来评估Disjoint-FVS问题实例的复杂程度(改变使用单个数值来评估的方式)。
这在以前的参数算法文献中,也是十分罕见的。本节中,我们将给出具体的评估方法以及对其边界的证明。

首先,得益于前人的工作成果\upcite{cao2010feedback},我们知道某些情况下Disjoint-FVS问题是存在多项式解法的,
对于这部分满足条件的顶点,我们并不需要对进行枚举,而是可以在枚举完所有其他顶点后,直接在多项式时间内求解剩余问题。
下面,我们引入其中一个算法,如下:
\begin{lemma} \label{Disjoint-FVSSolvable}
对于一个Disjoint-FVS问题实例$(G, U, D, k)$,如果所有的顶点$v \in D$,满足其邻居数量$\abs{N(v)} \le 3$,
那么存在一个时间复杂度为$O(n^2log^6n)$的多项式算法,或者构造出图G的反馈顶点集$X$满足$\abs{X} \le k$并且$X \subseteq D$,
或者回答不存在这样的反馈顶点集。
\end{lemma}

定义顶点集合$T = \{v \in D\;|\;\abs{N(v)} = 3 \text{ 并且 } N(v) \subseteq U\}$。
根据引理\ref{Disjoint-FVSSolvable},我们并不需要对顶点集合T中的顶点进行枚举。
接下来,给出一个引理对顶点集合T的大小进行分析。

\begin{lemma} \label{Disjoint-FVSLimit}
对于Disjoint-FVS问题实例$I(G, U, D, k)$,如果$k + \frac{cc(G[U])}{2} - \abs{T} < 0$那么该实例肯定没有符合条件的解。
\end{lemma}

\begin{proof}
我们通过证明该命题的逆反命题,来证明引理的正确性。

假设顶点集合X是图G的一个反馈顶点集合,并且$X \subseteq D, \abs{X} \le k$,即问题实例I存在符合条件的解X。
则对于任意顶点$v \in (T \setminus X)$,v连接了3个不同的$G[U]$的连通块,所以$cc(G[U]) \ge 2\abs{T\setminus X} + 1$。
结合$\abs{X} \le k$, 可以获得,$\abs{T} = \abs{X \cap T} + \abs{T \setminus X} < k + \frac{cc(G[U])}{2}$。

证毕。
\end{proof}

\begin{definition}
对于Disjoint-FVS问题实例$I(G, U, D, k)$,定义其评估函数如下,
\[\mu(I) = (\mu_1(I), \mu_2(I)) = (k, k + \frac{1}{2}cc(G[U]) - \abs{T})\]
\end{definition}

显然,对于该评估函数,当$\mu_1(I) \le  0$或者$\mu_2(I) < 0$的时候,问题实例多项式可解。
当$\mu_2(I) < 0$时,根据引理\ref{Disjoint-FVSLimit},直接返回不存在符合条件的解;
当$\mu_1(I) = 0$时,那么检查图G是否森林就可以立刻返回答案。

\subsection{收缩规则}
本节中,我们给出若干特殊情况下,应用在Disjoint-FVS问题实例上的收缩规则,并且证明他们是安全且有效的。
假设对于一个收缩规则,将问题实例$I$转化成问题实例$I'$,我们说该规则是安全的,代表新问题实例$I'$有解当且仅当原问题实例$I$有解;
说该规则是有效的,则意味着其两个评估参数都没有上升,即$u_1(I') \le u_1(I), u_2(I') \le u_2(I)$。 \\

\begin{reducerule}
移除图G中所有度数为1的顶点。
\end{reducerule}

\begin{reducerule}
如果存在一个顶点$v \in D$,其有两个邻居在图$G[U]$的同一个连通块中,则移除顶点v并令k减少1, 即令$G \leftarrow G \setminus v, k \leftarrow k - 1$。
\end{reducerule}

\begin{reducerule}
如果存在一个顶点$v \in D$,其度数为2且至少有一个邻居在$U$中,则将其从顶点集合D中移到顶点集合U中,即令$U \leftarrow U \cup \{v\}, D \leftarrow D \setminus \{v\}$。
\end{reducerule}

\begin{reducerule}
如果存在一个顶点$v \in D$,其度数为2,则其删除并且在它的两个邻居间连一条边,即令$G \leftarrow G \setminus \{v\}, E(G) \leftarrow E(G) \cup N(v)$。
\end{reducerule}


\begin{reducerule}
如果存在一个度数为3的顶点$v \in D$,满足$\abs{D \cap N(v)} = 1$,
假设其3个顶点分别为$w \in D, u_1 \in U, u_2 \in U$,
删除顶点$v$和$w$之间的边,添加新顶点$z$属于顶点集合$U$,并且连接其与顶点$w$和$v$,
即令$Z \leftarrow Z \cup \{z\}, E(G) \leftarrow E(G) \cup \{e_{zw}, e_{zv}\} \setminus \{e_{wv}\}$。
\end{reducerule}


对于每个问题实例,我们按照给出顺序依次应用这个五个收缩规则,直到没有规则适用。
显然,这五条收缩规则都可以在多项式时间内应用,并且由于每次应用都会导致问题实例的规模的收缩,因此执行次数是$O(n)$级别的,故整个过程也是多项式时间的。
接下来,我们证明他们都是安全且有效的。

\begin{lemma}
收缩规则 1-5都是安全且有效的。
\end{lemma}
\begin{proof}
对于收缩规则1,度数为1的顶点不可能存在于任何环中,可以直接忽略。
对于收缩规则2,如果不将顶点$v$取进反馈顶点集中,则剩余图必然存在环。
因此规则1和2都是安全的。

而在规则3、4中,我们可以证明假设存在一个满足问题实例$I(G, U, D, k)$的反馈顶点集X包含顶点$v$,那么一定可以构造另外一个符合条件的反馈顶点集$X'$不包含顶点v。
假设图$G \setminus (X \setminus \{v\})$中存在环(否则可以令$X' = X \setminus \{v\}$),
因为图$G \setminus X$不存在环,所以图$G \setminus (X \setminus \{v\})$中的环肯定经过顶点$v$,
又因为顶点$v$度数为$2$且其邻居不属于同一个$G[U]$连通块(否则会应用收缩规则2),所以环仅有一个并且环上一定有顶点不属于$U$,我们取其中一个记为$u$。
构造$X' = X \setminus \{v\} \cup \{u\}$,显然符合要求。因此规则3和4都是安全的。

对于这4条规则,显然k和$cc(G[U])$都不会上升。而集合T缩小只可能发生在规则1删除一个属于U的顶点时,并且此时$cc(G[U])$也会减少1。因此这4条规则都是有效的。

最后讨论收缩规则5,收缩规则5使得新的问题实例增加了一个顶点$z \in U$,然而并没有增加任何新的环。
可以观察到如果新图上的环经过顶点$z$,那么这个环在原图上也存在,其经过了边$e_{wv}$。
因此,在应用规则5前后的两个问题实例上求解反馈顶点集实际上是等价的,即规则5是安全的。
观察收缩规则5在创造一个新的图$cc(G[U])$的连通块的同时,令顶点$v$满足$\abs{N(v)} = 3$并且$N(v) \subseteq U$,因此顶点集合$T$也增加了1。
故,应用完收缩规则5有,$u_a(I') = u_a(I) = k, u_b(I') = u_b(I) + 1/2 - 1 = u_b(I) - 1/2$。因此收缩规则5也是安全有效的。

证毕。
\end{proof}


\subsection{分支规则}
前文已经提到,我们的算法是基于分支搜索的,每次会从顶点集合D中取出一个顶点,通过假设它存在于反馈顶点集或者不存在来获得两个搜索分支。
为了获得更加优秀运行时间上界,我们希望通过选择最合适的顶点,以提高评估函数$\mu(I)$的下降程度。
本节中我们首先给出合适的选择顶点的方法,并且对应用分支规则后的两个问题实例的评估函数变化进行分析。

\begin{lemma} \label{vertexNDge3}
对于Disjoint-FVS问题实例$I(G,U,D,k)$,当收缩规则1-5都不能应用且顶点集合$D \setminus T$非空时,
至少存在一个顶点$v \in (D \setminus T)$,其邻居中至少有3个顶点在集合U。
\end{lemma}
\begin{proof}
首先,收缩规则1-5都不能应用可以推导出D中所有顶点度数都大于等于3。
其次顶点集合$D \setminus T$的导出子图是森林,取森林的任意孤立节点或者叶子结点v,顶点v满足$\abs{N(v) \cap D} \le 1$。
再结合顶点v不满足收缩规则5的应用条件,可以得知顶点v至少有3个邻居在U中。
\end{proof}

根据上述引理,我们总能应用以下分支规则。\\

\begin{tabular}{ p{0.9\headwidth} }
  \hline
  \textbf{分支规则}\\
  \textbf{前提:} 存在一个顶点$v \in (D \setminus T)$,其邻居中至少有3个顶点在集合U内。\\
  \textbf{分支一:} $(G_1 \leftarrow G \setminus \{v\},\; U_1 \leftarrow U,\; D_1 \leftarrow D \setminus \{v\}, k_1 \leftarrow k - 1)$\\
  \textbf{分支二:} $(G_2 \leftarrow G,\; U_2 \leftarrow U \cup \{v\},\; D_2 \leftarrow D \setminus \{v\}, k_2 \leftarrow k)$\\
  \hline
\end{tabular} \vspace{0.5cm}

对于分支一,我们将顶点v取进了反馈顶点集,即可以获得一个不包含顶点v的新问题实例$I_1$,此时$k_1$减少1,同时$cc(G_1[U_1])$和$\abs{T_1}$显然没有发生任何变化。
因此,
\begin{equation*}
  \begin{aligned}
    & u_a(I_1) = k - 1 = u_a(I) - 1 \\
    & u_b(I_1) = k - 1 + \frac{1}{2}cc(G[U]) - \abs{T} = u_b(I) - 1
  \end{aligned}
\end{equation*}
而对于分支二,当我们决定将顶点v排除出反馈顶点集时,可以将其移动到集合U中,由于收缩规则2不能被应用,
顶点v的所有邻居处于不同的$G[U]$连通块中,因此$G[U_2]$依然是森林并且$cc(G[U_2]) \le cc(G[U]) - 2$。同时此时$k$保持不变,$\abs{T}$有可能增加但不可能减少。
因此有,
\begin{equation*}
  \begin{aligned}
    & u_a(I_2) = u_a(I) \\
    & u_b(I_2) \le k + \frac{1}{2}(cc(G[U]) - 1) - \abs{T} = u_b(I) - 1
  \end{aligned}
\end{equation*}

\subsection{复杂度分析}
对于Disjoint-FVS问题实例,假设函数$T(\mu(I))$表示其搜索树节点个数。
回顾上一节的分支规则,其产生两个搜索分支,我们已经分析了其评估函数的变化,因此对于当前问题实例的搜索树规模,有
\begin{equation*} \begin{aligned}
  T(\mu(I)) & = T(\mu_a(I), \mu_b(I)) \\
            & \le T(\mu_a(I) - 1,\; \mu_b(I) - 1) +  T(\mu_a(I),\; \mu_b(I) - 1)
\end{aligned} \end{equation*}
此外,之前也讨论了当$\mu_a \le 0$或者$\mu_b \le 0$时,$T(\mu) = 1$。因此,我们可以得知
\[T(\mu(I)) \le C^{\mu_a}_{\mu_b} \le C^k_{k + cc(G[U])/2}\]
另外在搜索树上应用收缩规则和分支规则的时间复杂度都是多项式的,
因此对于任何Disjoint-FVS问题实例,可以在$\mathcal{O}^*(C^k_{k + cc(G[U])/2})$时间复杂度内构成出符合条件的反馈顶点集合或者回答不存在这样的反馈顶点集合。
至此我们成功证明了引理\ref{Disjoint-FVSTime}。

