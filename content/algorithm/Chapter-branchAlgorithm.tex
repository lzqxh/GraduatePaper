\section{VCAL分支算法}
上一章中,已经给出了基于线性松弛下界的参数化顶点覆盖问题(VCAL)的核心化算法,
将图G在线性时间内收缩至满足全$\frac{1}{2}$函数是其LPVC唯一最优解(即$surplus(G) \ge 1$)。
本章之中,我们将给出3条分支规则作用于核心化后的图上,
这些规则组成了本章的分支算法。
同时我们保证每条规则都可以在线性时间内测试是否适用并执行。
最后证明在这些分支规则的共同作用下整棵搜索树的大小不超过$2.618^\mu$。

为了更好的分析分支算法的时间负责度,首先给出以下引理:
\begin{lemma}
假设图G满足$surplus(G) \ge 1$,且对于其求解Dual-LPVC的网络W已经求得其最大流$f$。
令$G' = G \setminus S$,则可以在$O(\abs{S}(\abs{V} + \abs{E}))$的时间复杂度内
获得$G'$对应的网络$W'$上的最大流$f'$。
\end{lemma}

\begin{proof}
回顾\ref{TransformToNetwork}中内容,对于网络$W'$其有向图$G_{W'}$等价于有向图$G_W$移除顶点集合$L_S \cup R_S$以及相应的边集,即$G_{W'} = G_W \setminus \{L_S \cup R_S\}$。
因此如果网络W上一个可行流没有流量经过顶点集合$L_S \cup R_S$,则将其定义域限制为$E_{W'}$后便是网络$W'$上的可行流。

对网络W的最大流$f$进行退流操作,移除所有经过顶点集合$L_S \cup R_S$的流量,
因为在有向图$G_W$只有从$s$到$L_W$,从$L_W$到$R_W$以及从$R_W$到$t$的有向边,所以该操作可以在$O(\abs{E_W})$的时间复杂度内完成。将退流后的可行流定义域限定为$E_{W'}$,并命名为$f'$。显然$f'$是网络$W'$上的可行流。比较两个可行流流量,
可以得$F' \ge F - 2 \abs{S}$。

对网络$W'$的可行流$f'$进行增广,假设增广$\Delta$流量后$f'$成为最大可行流,
则增广的时间复杂度是$O(\Delta(\abs{E_W}))$。
因为网络$W'$中到达汇点t的边的容量和为$\abs{V'} = F - \abs{S}$,所以最大流流量$F' + \Delta = F - \abs{S}$。
整理可以获得$\Delta \le \abs{S}$。综上,将可行流$f'$增广至最大流的时间复杂度为$O(\abs{S}(\abs{V} + \abs{E}))$。

证毕。
\end{proof}

在分支算法开始之前,首先我们选择图G中任意一个顶点$v \in V$,
由于$surplus(G) \ge 1$,顶点$v$至少有两个邻居,取其中任意两个,记为$u_1, u_2$。
对于图G中的任何一个顶点覆盖,显然或者包含了顶点$v$, 或者同时包含了顶点$u_1, u_2$。
很自然,我们会考虑产生两个分支,使得其中一个将顶点$v$取进顶点覆盖集合,另外一个将顶点$u_1, u_2$取进。
为了限制搜索树大小,这里我们增加了一个前提条件。

\begin{branchrule}
前提:$vc^*(G \setminus \{u_1, u_2\}) = \abs{V} / 2 - 1 $
分支一, $(G \leftarrow (G \setminus v), k - 1)$
分支二, $(G \leftarrow (G \setminus \{u_1, u_2\}), k - 2)$
\end{branchrule}
