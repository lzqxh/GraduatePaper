\section{VCAL分支算法}
上一章中,已经给出了基于线性松弛下界的参数化顶点覆盖问题(VCAL)的核心化算法,
将图G在线性时间内收缩至满足全$\frac{1}{2}$函数是其LPVC唯一最优解(即$surplus(G) \ge 1$)。
本章之中,我们将给出3条分支规则作用于核心化后的图上,
这些规则组成了本章的分支算法。
对于每一条分支规则,我们保证其都可以在线性时间内测试是否适用,并且如果决定执行,执行时间复杂度也是线性的。
最后证明在这些分支规则的共同作用下整棵搜索树的节点数量不超过$2.618^\mu$。

\vspace{0.5cm}
为了更好的分析分支算法的时间负责度,首先给出以下引理:
\begin{lemma} \label{LemmaUpdateMaxflow}
假设图G满足$surplus(G) \ge 1$,且对于求解Dual-$LPVC(G)$模型的网络W已经求得其最大流$f$。
令$G' = G \setminus S$,则可以在$O(\abs{S}(\abs{V} + \abs{E}))$的时间复杂度内
获得$G'$上求解Dual-$LPVC(G')$模型的网络$W'$上的最大流$f'$。
\end{lemma}

\begin{proof}
回顾\ref{TransformToNetwork}中内容,对于网络$W'$其有向图$G_{W'}$等价于有向图$G_W$移除顶点集合$L_S \cup R_S$以及相应的边集,即$G_{W'} = G_W \setminus \{L_S \cup R_S\}$。
因此如果网络W上一个可行流没有流量经过顶点集合$L_S \cup R_S$,则将其定义域限制为$E_{W'}$后便是网络$W'$上的可行流。

对网络W的最大流$f$进行退流操作,移除所有经过顶点集合$L_S \cup R_S$的流量,
因为在有向图$G_W$只有从$s$到$L_W$,从$L_W$到$R_W$以及从$R_W$到$t$的有向边,所以该操作可以在$O(\abs{V} + \abs{E})$的时间复杂度内完成。
将退流后的可行流流量函数定义域限定为$E_{W'}$,并命名为$f'$。显然$f'$是网络$W'$上的可行流。比较两个可行流的流量,
可以得$F' \ge F - 2 \abs{S}$。

对当前网络$W'$的可行流$f'$使用Ford-Fulkerson算法进行增广,假设增广$\Delta$流量后$f'$成为最大可行流,
则增广的时间复杂度是$O(\Delta(\abs{V} + \abs{E}))$。
因为网络$W'$中到达汇点t的边的容量和为$\abs{V'} = F - \abs{S}$,所以最大流流量$F' + \Delta \le F - \abs{S}$。
整理可以获得$\Delta \le \abs{S}$。综上,将可行流$f'$增广至最大流的时间复杂度为$O(\abs{S}(\abs{V} + \abs{E}))$。

证毕。
\end{proof}


\subsection{分支策略}
在分支算法开始之前,首先我们在图G中任意取一个顶点$v \in V$,
由于$surplus(G) \ge 1$,顶点$v$至少有两个邻居,取其中任意两个,记为$u_1, u_2$。


对于图G中的任何一个顶点覆盖,显然或者包含了顶点$v$, 或者同时包含了顶点$u_1, u_2$。
很自然,我们会考虑产生两个搜索分支,使得其中一个将顶点$v$取进顶点覆盖集合,另外一个将顶点$u_1, u_2$取进。
为了限制搜索树大小,这里我们增加了一个前提条件,
要求在图$G \setminus \{u_1, u_2\}$上全$\frac{1}{2}$函数是其Primal-$LPVC(G \setminus \{u_1, u_2\})$模型的最优解,
即$vc^*(G \setminus \{u_1, u_2\}) = (\abs{V} - 2) / 2$。\\

\begin{tabular}{ p{0.9\headwidth} }
  \hline
  \textbf{分支规则 一 }\\
  \textbf{前提:}  在图$G \setminus \{u_1, u_2\}$上,$vc^*(G \setminus \{u_1, u_2\}) = (\abs{V} - 2) / 2$\\
  \textbf{分支一:} $(G_1 \leftarrow G \setminus \{v\},\; k_1 \leftarrow k - 1)$\\
  \textbf{分支二:} $(G_2 \leftarrow G \setminus \{u_1, u_2\},\; k_2 \leftarrow k - 2)$\\
  \hline
\end{tabular} \vspace{0.5cm}

根据引理\ref{LemmaUpdateMaxflow},显然测试分支规则一的前提是否成立,更新分支一、分支二的问题实例的$LPVC$问题最优解均可以在线性时间内完成。

之后我们考虑分支规则一不被执行的情形,即发现$vc^*(G \setminus \{u_1, u_2\}) < (\abs{V} - 2) / 2$。
此时我们对于图$G \setminus \{u_1, u_2\}$执行\ref{KerneliseAlgorithm}节给出的核心化算法,
设$G'$为图$G\setminus\{u_1, u_2\}$收缩后得到的新图,且已经被核心化算法取进顶点覆盖集的顶点集合为$V_1 \subseteq V$,
已经被标记不属于定点覆盖集合的顶点集合为$V_0 \subseteq V$。

\begin{claim}
对于顶点集合$V_1$, $V_0$有
\begin{itemize}
  \item{(1)}$\abs{V_1} = \abs{V_0} - 1$;
  \item{(2)}顶点集合$V_0 \subseteq V$是图G的一个独立集,且$N_G(V_0) = V_1 \cup \{u_1, u_2\}$。
\end{itemize}
\end{claim}
\begin{proof}
(1)构造一个图$G\setminus\{u_1, u_2\}$上Primal-$LPVC(G\setminus\{u_1, u_2\})$模型的解$ : V \setminus \{u_1, u_2\} \rightarrow \{0, \frac{1}{2}, 1\}$如下,
\begin{equation*}
   \begin{cases}
            x^*_v =0, & \mbox{对于顶点$v \in V_0$;}  \\
            x^*_v =1, & \mbox{对于顶点$v \in V_1$;}  \\
            x^*_v =\frac{1}{2}, & \mbox{其他。}
          \end{cases}
\end{equation*}
由\ref{KerneliseAlgorithm}节给出的核心化算法定义,显然这样构造出来的$x^*$是Primal-$LPVC(G\setminus\{u_1, u_2\})$的最优解。
又因为此时分支规则一不能应用,可得:
$vc^*(G\setminus\{u_1, u_2\}) = val(x^*) = \abs{V_1} + (\abs{V} - \abs{V_1} - \abs{V_0} - 2) / 2 < (\abs{V} - 2) / 2$,
化简得:$\abs{V_1} < \abs{V_0}$。

构造一个图$G$上Primal-$LPVC(G)$模型的解$x : V \rightarrow \{0, \frac{1}{2}, 1\}$如下,
\begin{equation*}
   \begin{cases}
            x_v =0, & \mbox{对于顶点$v \in V_0$;}  \\
            x_v =1, & \mbox{对于顶点$v \in V_1 \cup \{u_1, u_2\}$;}  \\
            x_v =\frac{1}{2}, & \mbox{其他。}
          \end{cases}
\end{equation*}
又因为已知全$\frac{1}{2}$函数是其图$G$上Primal-$LPVC(G)$唯一最优解,可得,
$val(x) = \abs{V_1} + 2 + (\abs{V} - \abs{V_1} - \abs{V_0} - 2) / 2 \ge vc^*(G) + 1/2 = (\abs{V} + 1) / 2$。
化简得:$\abs{V_1} \ge \abs{V_0} - 1$。

综上有$\abs{V_1} = \abs{V_0} - 1$。

(2)$V_0$在图$G\setminus\{u_1, u_2\}$中是独立集,显然其在图G中也是独立集。
假设$u_1$或者$u_2$不在$N_G(V_0)$中,则有$surplus(G) \le surplus_G(V_0) \le \abs{V_1} + 1 - \abs{V_0} = 0$。
这与已知$surplus(G) \ge 1$矛盾。故$u_1,u_2 \in N_G(V_0)$,即$N_G(V_0) = V_1 \cup \{u_1, u_2\}$。

证毕。
\end{proof}

后面根据$V_1 \cup \{u_1, u_0\}$是否图$G$中的独立集进行分类讨论。
%为了后文描述更加清晰,此处将$V_1 \cup \{u_1, u_0\}$重新定义成$V_1$。

在$V_1 \cup \{u_1, u_0\}$不是图$G$中的一个独立集的情况下,应用分支规则二如下:\\

\begin{tabular}{ p{0.9\headwidth} }
  \hline
  \textbf{分支规则 二 }\\
  \textbf{前提:}  顶点集合$V_1 \cup \{u_1, u_0\}\subseteq V$不是图$G$中的一个独立集\\
  \textbf{分支一:} $(G_1 \leftarrow G \setminus (V_0 \cup V_1 \cup \{u_0, u_1\}),\; k_1 \leftarrow k - \abs{V_1} - 2)$\\
  \hline
\end{tabular} \vspace{0.5cm}

分支规则二实际上没有产生多个搜索分支,但是为了更方便分析搜索树大小这里将其归入其中。首先我们给出一个引理来证明它的正确性。
\begin{lemma} \label{SurplusOne1}
对于图$G(V, E)$,假设$surplus(G) \ge 1$且存在独立集$S \subseteq V$满足$surplus(S) = 1$和$N(S)$并不是独立集,
则存在一个图$G$的最小顶点覆盖,其包含所有$N(S)$中顶点且不包含所有$S$中顶点。
\end{lemma}
\begin{proof}
令$G' = G[S \cup N(S)]$,假设$vc(G')\le \abs{S}$。
令顶点集合$VC'$是图$G'$的最小顶点覆盖集之一,将其分解为$VC' = A \uplus B$,其中$A \subseteq S$且$B \subseteq N(S)$。

因为$G'$中不存在孤立点且S是独立集,所以$S$中不被$A$包含的顶点肯定每个都与集合$B$相邻,即$N(S \setminus A) \subseteq B$。
由$N(S)$并非图G中的一个独立集,可以得知$B \neq \emptyset$,再考虑$\abs{A} + \abs{B} = \abs{VC'} \le \abs{S}$,
故$\abs{A} < \abs{S}$,因此$S \setminus A$显然也非空。
考虑独立集$S \setminus A$的盈余值$surplus(S \setminus A) \le \abs{B} - (\abs{S \setminus A}) \le 0$,这与$surplus(G) \ge 1$矛盾。
所以$G'$的最小顶点覆盖集大小大于$\abs{S}$。

又因为我们可以获得一个大小为$\abs{S} + 1$的顶点覆盖集合($N(S)$就是这样一个顶点集合),综上$vc(G') = \abs{S} + 1$。

通过上文的图$G'$的构造我们知道任意图$G$的最小顶点覆盖至少包含$\abs{S} + 1$个集合$S \cup N(S)$中的顶点。
假设顶点集合$VC$是图$G$的一个最小顶点覆盖集,我们可以构造一个新的顶点集合如下$VC^* = VC \setminus (S \cup N(S)) \cup N(S)$。
显然$VC^*$也会是图$G$的一个顶点覆盖集合,并且它的大小不会超过$VC$。
综上$VC^*$是图$G$的最小顶点覆盖,其包含所有$N(S)$中顶点且不包含所有$S$中顶点。证毕。
\end{proof}

再回顾分支规则二,首先在我们应用分支规则的图$G$上有$surplus(G) \ge 1$,其次$V_0$是图G上盈余值为1的一个独立集,
最后我们只在$V_0$的邻居顶点集合$V_1 \cup \{u_1, u_2\}$不是独立集的前提下应用分支规则二。
所以,根据引理\ref{SurplusOne1},此时存在图$G$的最小顶点覆盖集合包含所有$V_1 \cup \{u_1, u_2\}$中顶点,且不包含所有$V_0$中顶点。
至此成功证明了分支规则二的正确性,我们可以在不能应用分支规则一并且$V_1 \cup \{u_1, u_2\}$不是图$G$中的一个独立集的情况下,
先将$V_1 \cup \{u_1, u_2\}$中顶点选进顶点覆盖集合中。

接下来分析分支规则二的时间复杂度。
显然应用核心化算法在图$G \setminus \{u_1, u_2\}$上,
和判断顶点集合$V_1 \cup \{u_1, u_2\}$是否图G中的独立集均可以在$O(\abs{V} + \abs{E})$时间负责度内完成。
对于分支规则二产生的图$G_1$,其等价于前文中我们对图$G \setminus \{u_1, u_2\}$应用核心化算法收缩出来的新图$G'$,
在核心化算法中,已经获得了其对应的求解Dual-$LPVC(G')$模型的网络的最大流。所以不需要额外时间复杂度去求解该网络的最大流,
故分支规则二,也可以在线性时间内应用。

最后讨论剩余的情况,此时$V_1 \cup \{u_1, u_2\}$是图$G$中的一个独立集。

因为$surplus_G(V_1 \cup \{u_1, u_2\}) \ge surplus(G) \ge 1$,所以独立集$V_1 \cup \{u_1, u_2\}$至少有两个邻居不在点集$V_0$中,
即在图$G'$上,保留其至少两个邻居,这里我们取其中任意两个,设为$u_1', u_2'$。\\

\begin{tabular}{ p{0.9\headwidth} }
  \hline
  \textbf{分支规则 三 }\\
  \textbf{前提:} 顶点集合$V_1 \cup \{u_1, u_0\}\subseteq V$是图$G$中的一个独立集 \\
  \textbf{分支一:} $(G_1 \leftarrow G',\; k_1 \leftarrow k - \abs{V_1} - 2)$\\
  \textbf{分支二:} $(G_2 \leftarrow G' \setminus \{u_1', u_2'\},\; k_2 \leftarrow k - \abs{V_0} - 2)$\\
  \hline
\end{tabular} \vspace{0.5cm}

为了证明分支规则三的正确性,同样我们此处也引进一个新的引理,如下:
\begin{lemma} \label{SurplusOne2}
  对于图$G(V, E)$,假设$surplus(G) \ge 1$且存在独立集$S \subseteq V$满足$surplus(S) = 1$和$N(S)$是独立集,
则或者存在一个图$G$的最小顶点覆盖,其包含所有$N(S)$中顶点;
或者与之相反,存在一个图$G$的最小顶点覆盖其不包含所有$N(S)$中顶点。
\end{lemma}
\begin{proof}
令顶点集合$VC$是图$G$的最小顶点覆盖集之一,将其分解为$VC = A \uplus B \uplus C$,
其中$A \subseteq S$,$B \subseteq N(S)$且$C \cap (S \cup N(S)) = \emptyset$。
显然当$A = \emptyset$时,$B = N(S)$;当时$B = \emptyset$时,$A = S$。
因此我们只需要证明当顶点集合$A,B$同时非空时,可以构造出一新的图$G$的最小顶点覆盖使得其中一个集合为空。

假设顶点集合$A,B$同时非空。同理于引理\ref{SurplusOne1}的证明,可以得知此时$\abs{A} + \abs{B} \ge \abs{S} + 1$。
构造一新的顶点集合$VC' = N(S) \cup C$,显然$VC'$是图G的一个顶点覆盖,又因为$\abs{VC'} = \abs{C} + \abs{S} + 1 \le \abs{A} + \abs{B} + \abs{C} = \abs{VC}$,
所以它也是图G的一个最小顶点覆盖。

证毕。
\end{proof}

考虑在分支规则一与分支规则二均不能应用的情况下,
首先在我们应用分支规则的图$G$上已经有了$surplus(G) \ge 1$;
其次当分支规则一不能应用时候,$V_0$是图G上盈余值为1的一个独立集;
最后当分支规则二不能应用时候,$V_0$的邻居顶点集合$V_1 \cup \{u_1, u_2\}$是图$G$的一个独立集。
所以依据引理\ref{SurplusOne2},此时我们可以将原问题实例分成两个分支,
其中一个将顶点集合$V_1 \cup \{u_1, u_2\}$先取进顶点覆盖集,
另外一个将顶点集合$V_1 \cup \{u_1, u_2\}$的邻居先取进顶点覆盖集合。至此我们成功证明了分支规则三的正确性。

接下来我们分析执行分支规则三的时间复杂度。
根据我们的分类讨论情形,当分支规则一和二不能执行的时候,总会执行分支规则三,因此不需要额外测试是否执行该规则。
在利用核心化算法构造$G'$的时候,已经求得其Dual-$LPVC(G')$模型的最优解。
同时依据引理\ref{LemmaUpdateMaxflow},可以在求解Dual-$LPVC(G')$模型的网络的最大可行流的基础上,
在$O(2(\abs{V} + \abs{E}))$的时间负责度内构造分支二的对应的最大可行流。
故,分支规则三也能在线性时间复杂度内被执行。

\subsection{算法时间复杂度分析}
在上一节中,我们已经给出了3个分支规则,并且证明每个规则的正确性以及保证他们都能在线性时间内被执行。
接下来只剩下讨论搜索树大小。这里我们利用参数$u$的下降来进行分析。

对于我们核心化算法来说,在给出的时候已经证明其保证收缩后的VCAL问题实例$(G', k')$相比原来的VCAL问题实例$(G, k)$,有:
$\mu(G', k') \le \mu(G,  k)$。现在讨论每个分支规则应用后产生的每个分支的问题实例的参数,对比应用前的参数的变化。

\vspace{0.5cm}

首先回顾上一节给出的分支规则一。
我们通过或者使顶点覆盖集合包含顶点$v$, 或者使顶点覆盖集合包含顶点$u_1, u_2$,在原来VCAL问题实例的基础上产生了两个搜索分支。
这里对每个分支上的新问题实例的参数分别进行分析。

对于分支一,由于$surplus(G) \ge 1$,所以删除掉一个顶点后的图$G_1$上有$surplus(G_1) \ge 0$。
根据引理\ref{relationBwtSurplusAndLPVC},可以得知全$\frac{1}{2}$函数是图$G_1$上的Primal-$LPVC(G_1)$模型的最优解,即$vc^*(G_1) = \frac{\abs{V} - 1}{2}$。
所以有$\mu(G_1, k_1) = k_1 - vc^*(G_1) = k - 1 - \frac{\abs{V} - 1}{2} = \mu(G, k) - \frac{1}{2}$。

对于分支二,由分支规则一应用的前提可以得知$vc^*(G_2) = \frac{\abs{V} - 2}{2}$。
所以有$\mu(G_2, k_2) = k_2 - vc^*(G_2) = k - 2 - \frac{\abs{V} - 2}{2} = \mu(G, k) - 1$。

综上,这里更新分支规则一的描述,如下:\\

\begin{tabular}{ p{0.12\headwidth} | p{0.12\headwidth}p{0.66\headwidth} }
  \multicolumn{3}{l}{ \textbf{分支规则 一:} }\\
  \hline
  \textbf{前提:}  & \multicolumn{2}{l}{在图$G \setminus \{u_1, u_2\}$上,$vc^*(G \setminus \{u_1, u_2\}) = (\abs{V} - 2) / 2$}\\
  \hline
  \textbf{分支一:} & 问题实例:&$(G_1 \leftarrow G \setminus \{v\},\; k_1 \leftarrow k - 1)$ \\
                    & 参数:&$\mu(G_1, k_1) = \mu(G, k) - \frac{1}{2}$\\
  \hline
  \textbf{分支二:} & 问题实例:&$(G_2 \leftarrow G \setminus \{u_1, u_2\},\; k_2 \leftarrow k - 2)$\\
                    & 参数:&$\mu(G_2, k_2) = \mu(G, k) - 1$\\
  \hline
\end{tabular} \vspace{0.5cm}  

\vspace{0.5cm}
再考虑分支规则二。当分支规则一不能执行的时候,我们利用核心化算法对图$G \setminus \{u_1, u_2\}$进行收缩。
获得了算法收缩后的新图$G'$,被算法取进该图顶点覆盖集的顶点集合$V_1$以及被算法排除出该图顶点覆盖集合的顶点集合$V_0$。
而当顶点集合$V_1 \cup \{u_1, u_2\}$不是图G的独立集时候,我们应用分支规则二,产生了一个新的搜索分支。

对于分支规则二产生的唯一搜索分支,由于应用核心化算法得到的新图$G'$满足全$\frac{1}{2}$函数是图$G'$上的Primal-$LPVC(G')$模型的唯一最优解,
所以$vc^*(G_1) = \frac{\abs{V} - \abs{V_0} - \abs{V_1} - 2}{2}$。
另外已经证明了$\abs{V_1} = \abs{V_0} - 1$。故,
\begin{equation*} \begin{aligned}
    \mu(G_1, k_1) & = k_1 - vc^*(G_1)  \\
                  & = k_1 - \abs{V_1} - 2 - \frac{\abs{V} - \abs{V_0} - \abs{V_1} - 2}{2} \\
                  & = k_1 - (\abs{V_0} - 1) - 2 - \frac{\abs{V} - \abs{V_0} - (\abs{V_0} - 1) - 2}{2} \\
                  & = k - \frac{\abs{V}}{2} - \frac{1}{2} = \mu(G, k) - \frac{1}{2}
\end{aligned} \end{equation*}
更新分支规则二描述,如下;\\

\begin{tabular}{ p{0.12\headwidth} | p{0.12\headwidth}p{0.66\headwidth} }
  \multicolumn{3}{l}{ \textbf{分支规则 二:} }\\
  \hline
  \textbf{前提:}  & \multicolumn{2}{l}{顶点集合$V_1 \cup \{u_1, u_0\}\subseteq V$不是图$G$中的一个独立集}\\
  \hline
  \textbf{分支一:} & 问题实例:&$(G_1 \leftarrow G \setminus (V_0 \cup V_1 \cup \{u_0, u_1\}),\; k_1 \leftarrow k - \abs{V_1} - 2)$ \\
                    & 参数:&$\mu(G_1, k_1) = \mu(G, k) - \frac{1}{2}$\\
  \hline
\end{tabular} \vspace{0.5cm}  

\vspace{0.5cm}
最后看回分支规则三。当分支规则一和二不满足应用条件的时候,该分支规则会被应用,并且类似规则一,也会分裂出两个新的搜索分支。
对于分支一,同理于分支规则二的分支,可以得知$\mu(G_1, k_1) = \mu(G, k) - \frac{1}{2}$。
对于分支二,我们先证明以下断言:
\begin{claim}
全$\frac{1}{2}$函数是图$G_2$上Primal-$LPVC(G_2)$模型的最优解,即$surplus(G_2) \ge 0$。
\end{claim}
\begin{proof}
假设该断言不成立,即存在一个图$G_2$上的独立集满足$surplus_{G_2}(S) < 0$。
显然顶点集合$S \cup V_0 \subseteq V$,是图G上的一个独立集,
其在图G上的盈余值如下,
\begin{equation*} \begin{aligned}
  surplus_G(S \cup V_1) &\le surplus_{G_2}(S) + \abs{V_1}  - \abs{V_0} + 2 \\
                        &= surplus_{G_2}(S) + 1 \le 0
\end{aligned} \end{equation*}
\end{proof}

