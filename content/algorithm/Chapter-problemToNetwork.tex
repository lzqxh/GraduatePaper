
\section{将LPVC问题转换成最大网络流问题}

\subsection{原始对偶技巧}
基于线性规划问题的对偶性,所有纯粹线性规划问题存在对应的原始对偶模型。下面,我们将顶点覆盖问题进行线性松弛后的获得的LPVC问题(模型\ref{ModelLPVC1})转化成其原始对偶模型,如下:
\begin{equation} \label{ModelLPVC2} \begin{aligned}
  maximize\; & (\textbf{val(y)} = \sum_{e \in E}{y_e}) &\\
  subject\; to\; & \sum_{e \in \delta(v)}{y_e \le 1}, & v \in V \\
   & y_e \ge 0, & e \in E
\end{aligned} \end{equation}

这里为了区分两个求解LPVC问题的模型,我们将式子\ref{ModelLPVC1}命名为Primal-LPVC,将式子\ref{ModelLPVC2}命名为Dual-LPVC。
下面的引理给出两个模型之间的联系。
\begin{lemma} \label{relationBwtPrimalAndDual}
对于图$G(V, E)$,假设函数$x$是其Primal-$LPVC(G)$上的解,函数$y$是其Dual-$LPVC(G)$上的解,则$val(x) \ge val(y)$。
若$val(x) = val(y)$, 则x与y分别Primal-$LPVC(G)$和Dual-$LPVC(G)$上的最优解。
\end{lemma}
\begin{proof}
  对于任意E中的边$e = \{i, j\}$, 在Primal-$LPVC(G)$中有$x_i + x_j \ge 1$,在Dual-$LPVC(G)$中有$y_e \ge 0$。
  结合上述两个不等式,显然有,\[\sum\limits_{e=\{i, j\} \in E}{y_e(x_i + x_j)} \ge \sum\limits_{e \in E}{y_e}\]

  对任意V中的顶点v, $x_v \ge 0$ 且 $\sum_{e \in \delta(v)}{y_e} \le 1$,显然结合上述两个不等式可以获得,
  \[\sum\limits_{v \in V}{(\sum\limits_{e \in \delta(v)}{y_e})x_v} \le \sum\limits_{v \in V}{x_v}\]

  最后利用结合率,我们可以将$\sum_{e=\{i, j\} \in E}{y_e(x_i + x_j)}$整理成$\sum_{v \in V}{(\sum_{e \in \delta(v)}{y_e})x_v}$。
  综上有,
  \[
    \sum_{e \in E}{y_e} \le \sum_{e=\{i, j\} \in E}{y_e(x_i + x_j)}  = \sum_{v \in V}{(\sum_{e \in \delta(v)}{y_e})x_v} \le \sum_{v \in V}{x_v}
  \]

  另外当$\sum\limits_{v \in V}{x_v} = \sum\limits_{e \in E}{y_e}$时,对任意满足Primal-$LPVC(G)$约束条件的$x'$,
  我们可以获得$\sum\limits_{v \in V}{x'_v} \le \sum\limits_{e \in E}{y_e} = \sum\limits_{v \in V}{x_v}$。因此x是Primal-$LPVC(G)$的最优解,同理y也是Dual-$LPVC(G)$的最优解。
\end{proof}

\subsection{将Dual-$LPVC(G)$转化成最大网络流问题} \label{TransformToNetwork}
本节中我们首先将根据图$G(V, E)$构造一个网络W。
之后并通过证明任意一个网络W上的可行流与一组满足Dual-$LPVC(G)$约束的向量y一一对应且$F = 2val(y)$,
来证明求Dual-$LPVC(G)$的最优解等价与求解该网络上的最大流。
最后,我们给出通过最大可行流构造Primal-$LPVC(G)$与Dual-$LPVC(G)$最优解的方法。

求解Dual-$LPVC(G)$的网络定义如下:
\begin{definition}
对于图$G = (V, E)$, 定义与其对应的网络$W = (G_W,\; c)$,如下:
\begin{equation*}\begin{aligned}
    &G_W = (V_W, E_W) \\
    &V_W = \{l_v\;|\;v \in V\} \cup \{r_v\;|\;r \in V\} \cup \{ s, t \} \\
    &E_W = \{(s, l_v)\;|\;v \in V\} \cup \{(l_u, r_v)\;|\;\{u, v\} \in E\} \cup \{(r_v, t)\;|\;v \in V\}\\
    &c(e) =   \begin{cases}
        1, & \mbox{if } e = (s, l_v)\;or\;e = (r_v, t)\\
        \infty, & \mbox{otherwise.}\\
  \end{cases}
\end{aligned}\end{equation*}
\end{definition}

\begin{property}
图G上的一个满足Dual-$LPVC(G)$约束的函数y,对应网路W上的一个流量为$2val(y)$的可行流。
\end{property}
\begin{proof}
  首先,我们构造一个流量函数$f:E_W \rightarrow \mathbb{N}$, 如下:
  \begin{equation*}\begin{aligned} \begin{cases}
    f(s, l_v) = f(r_v, t) = \sum\limits_{e \in \delta(v)}{y_e}, & \mbox{for } v \in V \\
    f(l_u, r_v) = y_e, & \mbox{for } \{u, v\} \in E.
  \end{cases}\end{aligned}\end{equation*}
  回顾可行流需要满足的条件(式子\ref{EquationFlow}), 首先由Dual-$LPVC(G)$的约束可以得到,
  $0 \le f(s, l_v) = f(r_v, t) = \sum\limits_{e \in \delta(v)}{y_e} \le 1 = c(e)$。
  其次,对于任意$v \in V$ 显然,
  \[ \begin{aligned}
     \sum_{e\in \delta^+(l_v)}{f(e)} = f(s, l_v) = \sum\limits_{e \in \delta(v)}{y_e} = \sum_{\{v, u\} \in E}{f(l_v, r_u)} = \sum_{e\in \delta^-(l_v)}{f(e)} \\
    \sum_{e\in \delta^+(r_v)}{f(e)} = \sum_{\{v, u\} \in E}{f(l_u, r_v)}  = \sum\limits_{e \in \delta(v)}{y_e} = f(r_v, t) = \sum_{e\in \delta^-(r_v)}{f(e)}
  \end{aligned} \]
  综上$f$是网络W上的一个\textbf{可行流}, 其流量$F = 2\sum_{e \in E}{y_e} = 2val(y)$(每条边上的$y_e$被计入两次)。证毕。
\end{proof}

\begin{property}
网络W上的一个可行流$f$,对应一个图G上满足Dual-$LPVC(G)$约束的函数y,且$val(y) = \frac{F}{2}$。
\end{property}
\begin{proof}
构造y如下,$y_{e = \{u, v\}} = \frac{1}{2}(f(l_u, r_v) + f(l_v, r_u))$。
此时对于任意顶点$v \in V$, $\sum\limits_{e \in \delta(v)}{y_e} = \sum\limits_{\{u, v\}\in E}{\frac{1}{2}(f(l_u, r_v) + f(l_v, r_u))} \le \frac{1}{2}(c(s, l_v) + c(r_v, t)) = 1$。
因此y满足Dual-$LPVC(G)$约束。

另外显然$val(y) = \sum_{e \in E}{y_e} = \frac{1}{2}\sum\limits_{\{u, v\}\in E}{(f(l_u, r_v) + f(l_v, r_u))} = \frac{F}{2}$。证毕。
\end{proof}

最后我们给出通过网络W上的最大可行流,构造Primal-$LPVC(G)$最优解方法。

假设$f$是网络W上的最大可行流,则可以通过网络W在可行流$f$下的残余图$G_W(f)$构造Primal-$LPVC(G)$的半整数最优解x如下:
\begin{equation*}
  x_v = \begin{cases}
            0, & \mbox{如果在图$G_W(f)$中,$l_v$是从s出发可达的并且$r_v$是从s出发不可达的}  \\
            1, & \mbox{如果在图$G_W(f)$中,$l_v$是从s出发不可达的并且$r_v$是从s出发可达的}  \\
            \frac{1}{2}, & \mbox{其他。}
          \end{cases}
\end{equation*}

\textcolor[rgb]{1.00,0.00,0.00}{待补充,证明x是最优解}