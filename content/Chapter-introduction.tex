\chapter{绪论}
本章介绍了研究背景,相关研究现状,研究内容,意义,及论文的章节安排。

\section{研究背景及意义}
随着现代计算机科学的迅速发展,以及计算机科学在各个研究领域的应用,许多NP-hard问题大量的出现。
尽管在NP完全性理论\upcite{garey1979guide}中,NP是否等于P依然没有被证明,但是计算机界普遍的共识是,NP-hard问题时是无法用现代计算机有效求解的。
然而,所有这些应用中的NP难解问题是必须解决的问题,因此自NP完全性理论发展以来,人们提出了一系列实际解决NP难问题的方法,如
启发式算法\upcite{michalewicz2013solve}、近似算法\upcite{ausiello2012complexity,hochbaum1996approximation}、随机算法\upcite{motwani2010randomized,mitzenmacher2005probability}等。
但是,这些算法都存在着各自明显的不足。人们需要新的方法来解决这些NP-hard问题。
参数计算与复杂性理论正式在这种情况下诞生的。
二十世纪末,以DowneyR和FellowsM为代表的计算机理论专家首次提出了参数复杂性的概念,
并在其专著《参数复杂性》\upcite{downey1998parameterized}中系统地阐述了参数计算与复杂性理论,标志着参数计算与复杂性理论的诞生。
自从参数计算与复杂性理论的诞生以来的十几年的时间里,参数计算与复杂性理论得到迅速发展和壮大,很快成为理论计算机科学的一个重要分
支。

参数算法的研究起源于对很多计算问题的观察,研究人员发现其中很多问题都与一个与问题规模无关的关键参数k相联系。
而很多实际应用中,这一参数只在一个较小的区间内变化,与问题的规模并无直接关系。
从这个方面着手,很多理论上难解的问题可以在实际应用中得到解决。
首先,我们把初始NP-hard问题进行参数化,将原来的问题实例X,表示成其参数化版本$(X, k)$的形式。
当一个参数化的问题,可以在$O(f(k)n^c)$时间复杂度内解决,我们就称该问题是固定参数可解(Fixed-Parameterized Tractable,简称FPT)的。
这里c是某一常数,f(k)是一个只跟k有关的函数与问题的规模n无关。
一般情况下当k不是很大的时候,函数f(k)的值也不会很大。所以此时这一问题在实际应用中是可以有效求解的。
如在,在文献\cite{gonnet2000darwin}中,研究人员将NP-hard的基因序列冲突问题转化成参数化顶点覆盖问题。
而关于此问题已经存在了时间复杂度为$\mathcal{O}(1.2738^k)$的固定参数算法(Fixed-Parameterized Tractable Algorithm, 简称FPT算法)。
在基因序列问题的特殊生物背景决定了顶点覆盖集的规模k不超过60,而图的规模n可能非常大。因此此参数算
法在较短的时间内解决了基因序列冲突问题。

另外一方面,参数计算理论也为计算问题的复杂性提供进一步分析的工具。
如,经典的最大独立集问题,研究人员无法找到对于一个函数$f$,使得独立集问题可以在$O(f(k)n^c)$的时间复杂度内求解。
普遍认为,该问题是不可能是固定参数可解的。
为了描述问题的这一性质,人们引入了固定参数不可解的概念,称之为W[1]-hard。
在实际应用中,这方面的研究尽管没有具体解决问题,但是在改善问题的思路上提供了方向性的指引。
如,在文献\cite{papadimitriou1997complexity}中,Papadimitriou等证明了数据库查询在以询问语句规模(即语句中涉及变量数量)作为参数时候是固定参数不可解的。
这一结论的证明,让许多数据库研究人员避免消耗无谓的精力在一个不可行的方向中。


从上述例子中,我们可以看到参数计算与复杂性理论在解决实际应用中的问题具有独特的优势,
对其进行进一步丰富和发展正是非常有价值的研究方向。



\section{研究内容}
一般来说在参数计算理论的研究,主要可以分成以下三个方向: 
参数化问题的参数可计算性,参数可枚举性以及固定参数算法的设计。
其中参数化问题的可计算性主要集中在研究问题本身的复杂性,尝试证明证明问题是属于固定参数可解(FPT)还是属于W[1]-hard的;
固定参数可枚举性的研究是指在固定参数可解的基础上,研究能否有效的枚举问题的所有或者部分解;
而最后参数算法的设计,主要包括的参数算法技术的应用以及新的参数算法技术的研究。
本文主要工作是对一般无向图上基于线性松弛下界的参数化顶点覆盖问题和参数化反馈顶点集问题的参数算法的研究,
属于参数算法设计的范畴。

\subsection{基于线性松弛下界的参数化顶点覆盖问题的FPT算法研究}
顶点覆盖问题(Vertex Cover,缩写为VC),即对于参数k,询问给定图中是否存在大小不超过k的顶点集合满足:对于图中任意一条边的两个端点至少有一个在该集合中。
该问题是最早被提出来的21个NP-complete问题之一\upcite{karp1972reducibility},也是最早证明是固定参数可解(Fixed parameter tractable)的问题\upcite{downey2012parameterized}。

目前已经有很多这方面的研究,其中以k作为参数的固定参数算法已经可以运行在$\mathcal{O}(1.2738^k)$的时间复杂度内。
近年来有研究提出了顶点覆盖问题参数化的新思路,开始使用k与某些最小顶点覆盖问题的下界的差作为新的参数\upcite{narayanaswamy2012lp,razgon2009almost,mishra2011complexity,raman2011paths,lokshtanov2014faster,iwata2014linear}。
显然基于下界的参数化方法,有效地缩小了参数大小,比经典版本更加有实用价值。
本文对该问题的研究主要是基于线性松弛下界。
对于该问题的参数算法,最优秀的$f(\mu)$函数已经优化到$f(\mu) = 2.3146^\mu$。
然而尽管该算法获得了一个优秀的$f(\mu)$函数,但是算法本身多项式部分的级别十分高,实际执行效率并不高。
针对该问题,有研究人员期望能在保持c=1的同时尽量优化$f(u)$,当前最成功的是一个$O(4^{\mu}(\abs{V}+\abs{E}))$的线性固定参数算法。

我们在前人的研究基础上继续优化该问题,并成功获得时间复杂度为$O(2.618^{\mu}(\abs{V}+\abs{E}))$的线性固定参数算法,较之前的最好结果有了较大的改进。
\subsection{参数化反馈顶点集问题的FPT算法研究}
反馈顶点集问题(Feedback Vertex Set,简称FVS),是指给定参数k,询问给定图中是否存在大小不超过k的顶点集合满足:图中任何一个环都经过该顶点集合。
该问题也是最早21个NP-complete问题之一\upcite{karp1972reducibility}。

针对该参数化反馈顶点集问题的研究,分成了两条分支,
其一主要研究的是固定参数的随机算法,当前基于Cut\&Count技术,已经可以获得一个时间复杂度为$\mathcal{O}(3^k)$的固定参数随机算法\upcite{cygan2011solving};
另外的方向是设计确定性算法,当前最低的复杂度是Kociumaka等设计的$\mathcal{O}(3.592^k)$\upcite{kociumaka2014faster}。

本文中我们研究的是参数化反馈顶点问题的固定参数确定算法,我们成功设计了一个复杂度为$\mathcal{O}(3.598^k)$的FPT算法。
我们的算法与当前最快的算法,复杂度上十分接近,但是却简单直接很多(Kociumaka等的算法有12条分支规则之多,而我们的只有1条)。

\section{论文结构}
本论文共有五章,组织如下:

第一章:绪论。简要介绍固定参数算法的背景概念,并指明本文的主要工作,概述文章的组织结构。

第二章:相关问题研究现状。
这一章介绍了当前参数计算领域的研究现状,并重点介绍了顶点覆盖和反馈顶点集两个问题上今年来重要参数算法上的研究成果。

第三章:基于线性松弛下界的参数化顶点覆盖问题。
本章之中我们对基于线性松弛下界的参数化顶点覆盖问题进行了详细的分析,提出了一个时间复杂度为$O(2.618^{\mu}(\abs{V}+\abs{E}))$的线性固定参数算法,并给出正确性证明。

第四章:反馈顶点集问题的固定参数算法。
这一章里主要是讨论反馈顶点集,针对该问题提出了一个时间复杂度为$\mathcal{O}(3.598^k)$的固定参数算法,并给出正确性证明。

第五章:总结与展望。总结本文的工作内容,分析其中的不足之处,展望未来。
