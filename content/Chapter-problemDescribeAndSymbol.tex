\chapter{顶点覆盖问题的线性固定参数算法}
\section{引言}
顶点覆盖问题(Vertex Cover, 缩写为VC),即对于参数k,询问给定图中是否存在大小满足k的顶点集合满足:对于图中任意一条边的两个端点至少有一个在该集合中。
该问题是最早被提出来的21个NP-complete问题之一\cite{karp1972reducibility},也是最早证明是固定参数可解(Fixed parameter tractable)的问题\cite{downey2012parameterized}。

本章提出了一个时间复杂度为$O(2.618^{\mu}(\abs{V} + \abs{E}))$的线性固定参数算法。

\section{相关术语及问题的定义}
\subsection{有关图的一些术语的定义}
对于无向图$G=(V,E)$, V是图G的顶点集合, E是图G的边集合, 令$n = \abs{V}$表示图的顶点数量, $m = \abs{E}$表示图的边数量。
如果$e \in E$的两个端点分别为$u$和$v$,则使用集合$\{u, v \}$来表示边e。
对于图顶点的某个子集$V' \subseteq V$, 我们定义图G的关于$V'$的导出子图为$G' = (V', E')$(简记为$G[V']$), 其中$E' = \{\{u, v\} \in E| u \in V', v \in V' \}$。
设$v \in V$,定义由与v相邻的顶点构成的集合为$N_{G}(v) = \{ u \in V | \{u, v\} \in E\}$。
同理对于$V' \subseteq V$, 定义$N_{G}(V') = (\bigcup_{u \in V'}N_{G}(u)) \setminus V' $。
同时因为在某些情况下讨论$V'$的邻居集合时候我们不需要排除其本身中的顶点,
故额外定义$N_{G}[V'] = \bigcup_{u \in V'}N_{G}(u)$。
(在前后文没有歧义的情况下,我们会简称这三个集合为$N(v), N(V'),N[V']$)。

另外在图中所有与顶点v相邻的边的集合,我们称之为$\delta(v)$.

\begin{definition}[顶点覆盖集]
对于图$G(V, E)$, 若集合$V' \subseteq V$满足对于任意$\{u, v\} \in E$, $u \in V'$或者$v \in V'$, 则称$V'$是图G的一个顶点覆盖集。
若$V'$是图G所有顶点覆盖集中最小的一个,则称顶点覆盖集$V'$是图$G$的最小顶点覆盖集。这里我们约定用$vc(G)$表示图G的最小顶点覆盖集大小。
\end{definition}

\begin{definition}[独立集]
对于图$G(V, E)$, 若集合$V' \subseteq V$满足对于任意$u, v \in V'$, $\{u, v\} \notin E$, 则称$V'$是图G的一个独立集。
\end{definition}

设$V' \subseteq V$是图G的一个独立集,定义盈余值$surplus(V') = \abs{N(V')} - \abs{V'}$。同时对于整个图G其盈余值$surplus(G)$等于其所有独立集的盈余值的最小值。

对于有向图G(V, E), 若$e \in E$是从$u$到$v$的一条边,则使用元组$(u, v)$来表示边e。
设$v$是有向图G的一个顶点, 我们使用$\delta^-(v)=\{(u, v) \in E\}$来表示所有进入顶点v的边的集合,用$\delta^+(v)=\{(v, u) \in E\}$来表示所有从顶点v出发的边的集合。
同样我们将这两个定义扩展到有向图G中的顶点集合, 令$S \subseteq V$, 则$\delta^-(S)=\{(u, v) \in E | u \notin S, v \in S\}$表示所有进入S的边的集合,
$\delta^+(S)=\{(u, v) \in E | u \notin S, v \in S\}$表示所有离开S的边的集合。

\begin{definition}[强连通分量(strongly connected component)]
在有向图$G(V, E)$中,若集合$S \subseteq V$满足对于任意$u, v \in S$都存在一条路径可以从$u$到达$v$,则称S是有向图G中的一个强连通分量。
如果集合S同时满足$\delta^+(S) = \emptyset$, 则称S是有向图G的一个尾强连通分量(tail strongly connected component)。
\end{definition}

在文献\cite{tarjan1972depth,sharir1981strong}中,已经分别给出了著名的Tarjan算法和Kosaraju算法。
这两个算法都能在线性时间复杂度(即$O(n + m)$)内求得有向图的所有强连通分量,包括尾强连通分量。
因为他们都是计算机领域非常重要且经典的算法,本文默认读者都已经对其有一定了解,后续不再进行展开,而是直接使用结论:\textbf{求有向图的尾强连通分量可以在线性时间复杂度($O(n + m)$)内完成}。

接下来我们给出网络及其流的定义。

在图论中,网络是指一个边带权有向图,即有向图$G(V, E)$加上一个边的权值函数$c: E \rightarrow \mathbb{N}^+$(在网络中又称容量函数)。
对于V中的两个顶点$s, t \in V$,一个可行s-t流(在上下文没有歧义的情况下,简称可行流),是指一个从边集到正整数集的映射$f: E \rightarrow \mathbb{N}^+$ 满足:
\begin{equation} \label{EquationFlow} \begin{aligned}
  \sum_{e\in \delta^+(s)}{f(e)}&=\sum_{e\in \delta^-(t)}{f(e)}& \\
  \sum_{e\in \delta^+(v)}{f(e)}&=\sum_{e\in \delta^-(v)}{f(e)}, & v &\in V \setminus \{s, t\} \\
  0 \le f(e) &\le c(e), & e &\in E \\
\end{aligned} \end{equation}
一般我们用$F=\sum_{e\in \delta^+(s)}{f(e)}$来表示可行流的流量。
最后,网络$(G, c)$在流量$f$下的残余图,我们用有向图$G(f)$来表示,其中:
\begin{equation*} \begin{aligned}
  G(f) = &(V, E(f)) \\
  E(f) = &\{(u,v)\;|\;(u, v) \in E\;and\;f(u, v) < c(u, v)\} \cup \\
  &\{(v,u)\;|\;(u, v) \in E\;and\;f(u, v) > 0\}
\end{aligned} \end{equation*}

在文献\cite{ford1962flows}, 已经给出著名的Ford-Fulkerson算法。本文也不打算对其进行展开,这里直接给出结论:\textbf{使用Ford-Fulkerson算法可以在$O(F(n + m))$时间复杂度内计算一个流量为F的s-t可行流}。

\subsection{顶点覆盖问题的数学描述及其线性规划松弛}
对于给定图$G(V, E)$, 求其最小顶点覆盖集合可以表示成下面整数规划模型:
\begin{equation} \label{ModelLPVC1} \begin{aligned}
  minimize\; & (\textbf{val(x)} = \sum_{v \in V}{x(v)}) &\\
  subject\; to\; & x(u) + x(v) \ge 1, &(u, v) \in E \\
   & x(u) \in \{0, 1\}, & u \in V
\end{aligned} \end{equation}
其中对于图的每一个顶点$v$, 当且仅当$x(v) = 1$时该顶点被选进顶点覆盖集合中。
显然上述整数规划模型的约束条件限制了对于图中的任意一条边中至少有一个点处于目标顶点覆盖集合的。
同时最小化目标函数$minimize\; \sum_{v \in V}{w(v)}$保证了该顶点覆盖集最小。这里我们用$vc(G)$表示最小顶点覆盖集合的大小。

在整数规划模型中,可以通过取消值必须是整数的限制进行线性规划松弛,可以获得对应线性规划模型。
这里我们将上面的模型中$x(u) \in \{0, 1\}$的条件进行松弛,
\begin{equation} \begin{aligned}
  minimize\; & (\textbf{val(x)} = \sum_{v \in V}{x(v)}) &\\
  subject\; to\; & x(u) + x(v) \ge 1, &(u, v) \in E \\
   & 0 \le x(u) \le 1, & u \in V
\end{aligned} \end{equation}

这里我们用$LPVC$来表示基于线性规划松弛的顶点覆盖问题,
用$vc^{*}(G)$来表示基于线性规划松弛的顶点覆盖问题的最优解的值。
显然,$vc(G) \ge vc^*(G)$。这里$vc^{*}(G)$是最小顶点覆盖的一个下界,我们称之为\textbf{线性松弛下界}。

如果某个函数$x:V \rightarrow \mathbb{R}$, 满足上述模型中的所有约束条件,则x被称为$LPVC(G)$的一个解。
在所有解中,$val(x)$达到最大化要求的,就是$LPVC(G)$的最优解。
另外如果对于所有顶点$v \in V$,有$x(v) = \frac{1}{2}$,
则x被称为\textbf{全$\frac{1}{2}$函数},如果它同时也是$LPVC(G)$的解则被称为\textbf{全$\frac{1}{2}$解}。



早在19世纪80年代,纯粹的线性规划问题已经被证明是可以在多项式时间内求解的\cite{khachiian1979polynomial}。显然$vc^{*}(G)$也可以在多项式时间内获得。
这里尽管由于基于线性规划松弛的顶点覆盖问题中顶点的取值可以是分数,我们不能将$LPVC$的解直接转化成顶点覆盖问题的解,
但是松弛后的线性规划问题还是为原问题的近似算法以及固定参数算法的设计提供了很多有价值的参考。下面我们给出若干$LPVC$的已经被证明的性质。
\begin{lemma}[\cite{nemhauser1975vertex}]
对于图$G(V, E)$, 总是存在一个LPVC(G)的最优解$x$, 对于所有的$v \in V$ 满足$x(v) \in \{0, \frac{1}{2}, 1\}$。
我们称x为LPVC(G)的一个半整数最优解,如果对于所有的$v \in V$有$x(v) = \frac{1}{2}$, 则称x是LPVC的全$\frac{1}{2}$最优解。
\end{lemma}

在后续章节中因为我们总是处理LPVC(G)的半整数解,所以默认情况下我们说x是LPVC(G)的解是指x是LPVC(G))的半整数解。
同时为了更清晰的使用LPVC(G)的半整数解,我们约定对于$i \in \{0, \frac{1}{2}, 0\}$使用$V^x_i$表示集合$\{v \in V: x(v) = i\}$。

\begin{lemma}[\cite{nemhauser1975vertex}]\label{relationBwtVCAndLPVC}
如果x是LPVC(G)的最优解,那么存在一个图G的最小顶点覆盖集包含所有$V^x_1$中的顶点,并且不包含任何$V^x_0$中的顶点。
\end{lemma}

引理\ref{relationBwtVCAndLPVC}, 揭示了线性松弛后的问题与原问题的之间最直观的联系。后面我们会利用该引理来有效缩小原问题的规模。

\begin{lemma}[\cite{nemhauser1975vertex}]\label{relationBwtSurplusAndLPVC}
对于图$G(V, E)$,如果$surplus(G) \ge 0$,那么全$\frac{1}{2}$函数是$LPVC(G)$的一个最优解。
进一步看,如果$surplus(G) \ge 1$,那么全$\frac{1}{2}$是$LPVC(G)$的唯一最优解。
\end{lemma}

\subsection{基于线性松弛下界的参数化顶点覆盖问题}
在顶点覆盖问题中,因为$vc(G)\ge vc^*(G)$,显然不存在小于$vc^*(G)$的顶点覆盖集,换句话说$vc*(G)$是顶点覆盖问题的下界。
如果顶点覆盖固定参数算法中取所求顶点覆盖集大小k作为的参数,当$k < vc^*(G)$我们可以立刻返回“NO”,此时k是无意义的。
当且仅当$k \ge vc^*(G)$, 固定参数算法才是非平凡的。显然此处会很自然的考虑是不是所求顶点覆盖集大于$vc*(G)$的部分才有实际意义,能不能以$所求顶点覆盖集大小\; - vc*(G)$作为参数。
现给出基于线性松弛下界的顶点覆盖参数化问题(VERTEX COVER ABOVE LP, 简称VCAL)的完整定义。\\

\begin{tabular}{| p{0.9\headwidth} |}
  \hline
  基于线性松弛下界的参数化顶点覆盖问题(VCAL) \\
  \textbf{输入:} 图$G(V, E)$及整数k \\
  \textbf{参数:} $\mu = k - vc^*(G)$\\
  \textbf{问题:} 图G中是否存在一个顶点覆盖集合,其大小不超过k\\
  \hline
\end{tabular} \vspace{0.5cm}


在文献\cite{narayanaswamy2012lp}中,作者最早定义了VCAL问题,并且基于分支与界法获得一个$\mathcal{O}^*(2.618^{\mu})$的FPT算法。
在后续的研究中,Lokshtanov等将该问题的复杂度提升到了$\mathcal{O}^*(2.3146^{\mu})$ \upcite{lokshtanov2014faster}。
这两个算法都是非常巧妙的,然而尽管他们在问题指数部分复杂度获得了一个优秀的上界,但是算法本身因为多项式部分糟糕的级别,实际执行效率并不高。
针对该问题,Iwata等\upcite{iwata2014linear}尝试在缩小FPT算法指数部分的同时优化多项式部分复杂度,他们成功获得了一个时间复杂度为$O(4^{\mu}(\abs{V} + \abs{E}))$的线性固定参数算法。

我们将在前人的研究基础上继续优化该问题,并成功获得时间复杂度为$O(2.618^{\mu}(\abs{V} + \abs{E}))$的线性固定参数算法,较之前的最好结果有了较大的改进。
本章后面部分将给出具体算法及其证明分析。


\section{将LPVC问题转换成最大网络流问题}
本节中我们将会给出算法的具体流程。本文提出求解VCAL的算法主体部分是分支与界法,通过仔细设计分支规则以及分支之后的核心化技巧,
保证整棵搜索树大小不超过$2.618^(k-vc^*(G))$。我们借鉴Iwata等\upcite{iwata2014linear}核心化的思路,
在搜索树的每个节点也通过网络流来提升核心化的效率。最终成功获得多项式部分不超过$2.618^(k-vc^*(G))$的线性固定参数算法(FPT)。
\subsection{原始对偶技巧}
基于线性规划问题的对偶性,所有纯粹线性规划问题存在对应的原始对偶模型。下面,我们将顶点覆盖问题进行线性松弛后的获得的LPVC问题(模型\ref{ModelLPVC1})转化成其原始对偶模型,如下:
\begin{equation} \label{ModelLPVC2} \begin{aligned}
  maximize\; & (\textbf{val(y)} = \sum_{e \in E}{y_e}) &\\
  subject\; to\; & \sum_{e \in \delta(v)}{y_e \le 1}, & v \in V \\
   & y_e \ge 0, & e \in E
\end{aligned} \end{equation}

这里为了区分两个求解LPVC问题的模型,我们将式子\ref{ModelLPVC1}命名为Primal-LPVC,将式子\ref{ModelLPVC2}命名为Dual-LPVC。
下面的引理给出两个模型之间的联系。
\begin{lemma} \label{relationBwtPrimalAndDual}
对于图$G(V, E)$,假设函数$x$是其Primal-LPVC上的解,函数$y$是其Dual-LPVC上的解,则$val(x) \ge val(y)$。
若$val(x) = val(y)$, 则x与y分别是图G在Primal-LPVC和Dual-LPVC上的最优解。
\end{lemma}
\begin{proof}
  对于任意E中的边$e = \{u, v\}$, 在Primal-LPVC中有$x_u + x_v \ge 1$,在Dual-LPVC中有$y_e \ge 0$。
  结合上述两个不等式,显然有,\[\sum\limits_{e \in E}{y_e(x_u + x_v)} \ge \sum\limits_{e \in E}{y_e}\]

  对任意V中的顶点v, $x_v \ge 0$ 且 $\sum_{e \in \delta(v)}{ye} \le 1$,显然结合上述两个不等式可以获得,
  \[\sum\limits_{v \in V}{(\sum\limits_{e \in \delta(v)}{y_e})x_v} \le \sum\limits_{v \in V}{x_v}\]

  最后利用结合率,我们可以将$\sum_{e \in E}{y_e(x_u + x_v)}$整理成$\sum_{v \in V}{(\sum_{e \in \delta(v)}{y_e})x_v}$。
  综上有,
  \[
    \sum_{e \in E}{y_e} \le \sum_{e \in E}{y_e(x_u + x_v)}  = \sum_{v \in V}{(\sum_{e \in \delta(v)}{y_e})x_v} \le \sum_{v \in V}{x_v}
  \]

  另外当$\sum\limits_{v \in V}{x_v} = \sum\limits_{e \in E}{y_e}$时,对任意满足Primal-LPVC约束条件的$x'$,
  我们可以获得$\sum\limits_{v \in V}{x'_v} \le \sum\limits_{e \in E}{y_e} = \sum\limits_{v \in V}{x_v}$。因此x是Primal-LPVC的最优解,同理y也是Dual-LPVC的最优解。
\end{proof}

\subsection{将Dual-LPVC转化成最大网络流问题} \label{TransformToNetwork}
本节中我们首先将根据图$G(V, E)$构造一个网络W。
之后并通过证明任意一个网络W上的可行流与一组满足Dual-LPVC约束的向量y一一对应,来证明求Dual-LPVC的最优解等价与求解该网络上的最大流。
最后,我们给出通过最大可行流构造Primal-LPVC与Dual-LPVC最优解的方法。

\begin{definition}
对于图$G = (V, E)$, 定义与其对应的网络$W = (G_W,\; c)$,如下:
\begin{equation*}\begin{aligned}
    &G_W = (V_W, E_W) \\
    &V_W = \{l_v\;|\;v \in V\} \cup \{r_v\;|\;r \in V\} \cup \{ s, t \} \\
    &E_W = \{(s, l_v)\;|\;v \in V\} \cup \{(l_u, r_v)\;|\;\{u, v\} \in E\} \cup \{(r_v, t)\;|\;v \in V\}\\
    &c(e) =   \begin{cases}
        1, & \mbox{if } e = (s, l_v)\;or\;e = (r_v, t)\\
        \infty, & \mbox(otherwise.)\\
  \end{cases}
\end{aligned}\end{equation*}
\end{definition}

\begin{property}
图G上的一个满足Dual-LPVC约束的向量y,对应网路W上的一个流量为$2val(y)$的可行流。
text
\end{property}
\begin{proof}
  首先,我们构造一个流量函数$f:E_W \rightarrow \mathbb{N}$, 如下:
  \begin{equation*}\begin{aligned} \begin{cases}
    f(s, l_v) = f(r_v, t) = \sum\limits_{e \in \delta(v)}{y_e}, & \mbox{for } v \in V \\
    f(l_u, r_v) = y_e, & \mbox{for } \{u, v\} \in E.
  \end{cases}\end{aligned}\end{equation*}
  回顾可行流需要满足的条件(式子\ref{EquationFlow}), 首先由Dual—LPVC的约束可以得到,$0 \le f(s, l_v) = f(r_v, t) = \sum\limits_{e \in \delta(v)}{y_e} \le 1 = c(e)$。
  其次,对于任意$v \in V$ 显然,
  \[ \begin{aligned}
     \sum_{e\in \delta^+(l_v)}{f(e)} = f(s, l_v) = \sum\limits_{e \in \delta(v)}{y_e} = \sum_{\{v, u\} \in E}{f(l_v, r_u)} = \sum_{e\in \delta^-(l_v)}{f(e)} \\
    \sum_{e\in \delta^+(r_v)}{f(e)} = \sum_{\{v, u\} \in E}{f(l_u, r_v)}  = \sum\limits_{e \in \delta(v)}{y_e} = f(r_v, t) = \sum_{e\in \delta^-(r_v)}{f(e)}
  \end{aligned} \]
  综上$f$是网络W上的一个\textbf{可行流}, 其流量$F = 2\sum_{e \in E}{y_e} = 2val(y)$(每条边上的$y_e$被计入两次)。证毕。
\end{proof}

\begin{property}
网络W上的一个可行流$f$,对应一个图G上满足Dual-LPVC约束的向量y,且$val(y) = \frac{F}{2}$。
\end{property}
\begin{proof}
构造y如下,$y_{e = \{u, v\}} = \frac{1}{2}(f(l_u, r_v) + f(l_v, r_u))$。
此时对于任意顶点$v \in V$, $\sum\limits_{e \in \delta(v)}{y_e} = \sum\limits_{\{u, v\}\in E}{\frac{1}{2}(f(l_u, r_v) + f(l_v, r_u))} \le \frac{1}{2}(c(s, l_v) + c(r_v, t)) = 1$。
因此y满足Dual-LPVC约束。

另外显然$val(y) = \sum_{e \in E}{y_e} = \frac{1}{2}\sum\limits_{\{u, v\}\in E}{(f(l_u, r_v) + f(l_v, r_u))} = \frac{F}{2}$。证毕。
\end{proof}

最后我们给出通过网络W上的最大可行流,构造Primal-LPVC最优解方法。

假设$f$是网络W上的最大可行流,则可以通过网络W在可行流$f$下的残余图$G_f$构造Primal-LPVC的半整数最优解x如下:
\begin{equation*}
  x_v = \begin{cases}
            0, & \mbox{如果在图$G_f$中,$l_v$是从s出发可达的并且$r_v$是从s出发不可达的}  \\
            1, & \mbox{如果在图$G_f$中,$l_v$是从s出发不可达的并且$r_v$是从s出发可达的}  \\
            \frac{1}{2}, & \mbox{其他。}
          \end{cases}
\end{equation*}

\textcolor[rgb]{1.00,0.00,0.00}{待补充,证明x是最优解}

\section{VCAL问题核心化算法}
核心化技巧,是指通过多项式时间复杂度的算法将原本的参数化问题转化成一个规模更小(参数变小)的新问题,
期间需要保证两个问题是等价的(其中一个问题存在解等价于另外一个存在解) 。
这是固定参数(FPT)算法的设计过程中一个非常重要的技巧。
最直观的角度上看,通过核心化技巧问题规模缩小了,显然有利于后续的求解。
同时当问题实例已经到不能进一步核心化的状态,
那么可以从这一点上推导出当前实例很多特殊的性质。

本节中的核心化算法时间复杂度是$O(n + m)$,会在每次运行分支算法之前被调用,
以保证在分支算法运行的图中,全$\frac{1}{2}$函数是其LPVC唯一最优解($surplus(G) \ge 1$)。
另外,我们要求核心化算法的输入中已经包含了求解当前图的Dual-LPVC的网络W以及W上的最大可行流$f$,
并且算法会构造并输出好基于新图的网络以及相应最大可行流。
这个是我们本节里的核心化算法的目标。

接下来给出具体的算法,其包含两个阶段。

\subsection{核心化算法阶段一}
在本阶段中,算法对输入的图G进行收缩,使得全$\frac{1}{2}$函数成为图G的Primal-LPVC最优解
(根据引理\ref{relationBwtSurplusAndLPVC},即$surplus(G) \ge 0$)。
\vspace{0.5cm}

首先根据\ref{TransformToNetwork}节, 我们可以通过W上的最大可行流构造出图G的Primal-LPVC最优半整数解以及
Dual-LPVC最优半整数解,分别用$x$和$y$表示。
从引理\ref{relationBwtVCAndLPVC}上,我们知道存在一个图G的最小顶点覆盖包含所有$V^x_1$中的顶点,并且不包含
所有$V^x_0$中的顶点。因此在图G中可以首先将$V^x_1$里的顶点选进顶点覆盖集合中,
再判断图的剩余部分是否存在最多$k - \abs{V^x_1}$个顶点的顶点覆盖。
此外$V^x_0$中顶点,在$V^x_1$被取走后变成孤立点,也可以被删除,
所以我们从图G中删除掉所有$V^x_1$和$V^x_0$中顶点以及与这些顶点相邻的所有边,得到:
\begin{equation*}\begin{aligned}
    &G^* = (V^*, E^*),\;k^* = k - \abs{V^x_1} \\
    &V^* = V^x_{1/2} \\
    &E^* = \{\{u, v\} \in E\;|\;u \in V^*\;and\;v\in V^*\}
\end{aligned}\end{equation*}

我们已经成功将图G收缩成了图$G^*$,接下来要考虑如何在快速构造出图$G^*$的LPVC最优解。
因为图$G^*$本质上是图G在顶点子集$V^x_{1/2}$上的导出子图,很自然的我们会猜想,
是不是在函数$x,y$定义域收缩为$V^*$和$E^*$就分别是图$G^*$的Primal-LPVC, Dual-LPVC最优解。
接下来,我们对该猜想进行证明。

将函数$x,y$定义域分别限定为$V^*$以及$E^*$,
获得函数$x^*:V^* \rightarrow \mathbb{R}$及$y^*:E^* \rightarrow \mathbb{R}$,
即对任意图$G^*$上的顶点v,有$x^*_v=x_v$;对任意图$G^*$上的边e,有$x^*_e=x_e$。
\begin{claim}
$x^*, y^*$分别是图$G^*$的Primal-LPVC, Dual-LPVC半整数最优解。
\end{claim}
\begin{proof}
因为$x, y$分别是图G的Primal-LPVC, Dual-LPVC半整数最优解,
显然$x^*, y^*$也应该分别是图$G^*$的Primal-LPVC, Dual-LPVC半整数解。
否则在图$G^*$上存在线形规划模型约束条件不满足,而该条件在图$G$上也依然不满足,这与$x,y$作为最优解冲突。

根据定义可以知道,$x^*$是从$x$中移出掉$V^x_1$和$V^x_0$中顶点后获得的,故$val(x) - \abs{V^x_1} = val(x^*)$。
另外,$y^*$相比$y$移除了所有与$V^x_1$相邻的边,即$val(y) - val(y^*) = \sum_{e \in (E \setminus E^*)}{y_e}
\le \sum_{v \in V^x_1}{\sum_{e \in \delta(v)}{y_e}}\le\abs{V^x_1}$。
结合, $val(y) = val(x)$, 可知$val(y^*) \ge val(x^*)$。

因此$x^*, y^*$分别是图$G^*$的Primal-LPVC, Dual-LPVC解,且$val(y^*) \ge val(x^*)$,
根据引理\ref{relationBwtPrimalAndDual}有$x^*, y^*$分别是图$G^*$的Primal-LPVC, Dual-LPVC最优解。

证毕。
\end{proof}

比较两个问题实例$(G, k), (G^*, k^*)$,根据引理\ref{relationBwtVCAndLPVC},可以很容易判断他们是等价的。
对于现在新的VCAL上的问题实例$(G^*, k^*)$显然其参数相比于原实例$(G, k)$保持了不变,如下:
\begin{equation*} \begin{aligned}
  \mu(G^*, k^*) &= k^* - vc^*(G^*) \\
                &= (k - \abs{V^x_1}) - (vc^*(G) - \abs{V^x_1}) \\
                &= k - vc^*(G) = \mu(G, k)
\end{aligned} \end{equation*}

同时既然我们已经有了关于图$G^*$的Dual-LPVC最优解,
那么显然也可以在线性时间复杂度内构造出网络$W^*$的最大可行流。

\subsection{核心化算法阶段二}
因为在上一节中已经将VCAL问题实例$(G, k)$收缩至使得全$\frac{1}{2}$函数是图$G$的LPVC最优解(不保证是唯一最优解),
并且构造出对应的网络$W$的最大流$f$。
接下来在本阶段内,我们会对其问题进行进一步核心化,使得全$\frac{1}{2}$函数成为图$G$的LPVC\textbf{唯一半整数最优解}。

首先,我们给出该阶段算法的基本思路。

如果在当前图G满足$surplus(G) = 0$,根据定义可以找到图G的某个独立集$S \subseteq V$,使得$surplus(S) = 0$。
构造函数$x':V \rightarrow \mathbb{R}$如下,对于$N(S)$中顶点令$x'_v = 1$; 对于$S$中顶点令$x'_v = 0$; 其他顶点$x'_v = \frac{1}{2}$。显然$x'$也是$G$的LPVC最优解。
类似上一节,我们可以通过将$V^{x'}_1$(即$N(S)$)中顶点选进顶点覆盖集,来将问题实例从$(G,k)$收缩至$(G[V\setminus(S \cup N(S))], k - \abs{N(S)})$。
重复寻找满足条件的独立集,收缩,直到在当前图G中$surplus(G) > 0$,则此时已经达到我们核心化的目标,全$\frac{1}{2}$函数已经是图$G$的LPVC唯一半整数最优解。

观察上诉思路,我们发现将算法改进到线性时间复杂度的瓶颈在于如何快速的找到图G中满足$surplus(S) = 0$的独立集$S$。
为了突破该瓶颈,我们需要借助网络$W$在最大可行流$f$下的残余图$G_W(f)$。
下面给出若干引理来揭示网络$W$在最大可行流$f$下的残余图$G_W(f)$与满足条件的独立集之间的关系。

回忆网络W的构造,我们将原本图G中的顶点v,拆分成$l_v$和$r_v$两个顶点。
为了描述的方便,此处引入几个新的集合的定义,假设$S_W \subseteq V_W$是网络W中图$G_W$的顶点子集,
令$S_L = \{v\;|\;l_v \in S_W\},S_R = \{v\;|\;r_v \in S_W\}$。
注意此处,$S_W$是网络W中的顶点集合,而$S_L, S_R$是图G中的顶点集合。
令$L_W = \{l_v \in V_W\}, R_W = \{r_v \in V_W\}$。假设$V' \subseteq V$是图G中顶点集合,
另$L_{V'} = \{l_v\in V_W\;|\;v \in V'\},R_{V'} = \{r_v\in V_W\;|\;v \in V'\}$。
注意此处,$L_W, R_W, L_{V'}, R_{V'}$均是网络W中的顶点集合。

\begin{lemma} \label{residualgraph1}
对于网络W中图$G_W$的顶点子集$S_W \subseteq (L_W \cup R_W)$, 以下两个条件等价:
(1)在网络W的残余图中$G_W(f)$中不存在从$S_W$到$(L_W \cup R_W) \setminus S_W$的边;
(2)$N_G[S_L] = S_R$ 并且 $\abs{S_L} = \abs{S_R}$。
\end{lemma}
\begin{proof}
  对于当前的图G,我们知道$vc^*(G) = \abs{V} / 2$,网络W最大可行流$f$的流量$F=2vc^*(G)=\abs{V}$。
  所以对于图G的任意顶点$v \in V$,显然有$f(s, l_v) = f(r_v, t) = 1$。
  假设$V' \subseteq V$,可得流入$L_{V'}$的流量等于流出$R_{V'}$的流量等于$\abs{V'}$。
  回顾残余图的定义,其边有正向反向两种,这里我们也将$S_W$到$(L_W \cup R_W) \setminus S_W$的边分成两个集合,如下:
  $E_L = \{(u, v) \in E_W\;|\;u \in S_W,v \in (L_W \cup R_W) \setminus S_W\}$,
  $E_R = \{(v, u)\;|\;(u, v) \in E_W,f(u, v) > 0,u \in (L_W \cup R_W) \setminus S_W,v \in S_W\}$。

  $(1)\rightarrow (2)$: 假设条件(1)成立,等价于$E_L = E_R = \emptyset$。
  由$E_L = \emptyset$,可以知道所有$S_L$中顶点的邻居都在$S_R$中,既$N_G[S_L] \subseteq S_R$。
  由$E_R = \emptyset$,可以知道没有从$S_W$外的顶点有流量流入$S_W \cap R_W$中,所以从$S_W \cap R_W$流出的流量应该小于等于流入$S_W \cap L_W$中的流量,即$\abs{S_R} \le \abs{S_L}$。
  最后,从$S_W \cap L_W$流出的流量显然不可能超过从$R_{N_G[S_L]}$流到汇点t的流量,即$\abs{S_L} \le \abs{N_G[S_L]}$。
  综上有,$\abs{S_L} = \abs{N_G[S_L]} = \abs{S_R}$,再结合前面得到的$N_G[S_L] \subseteq S_R$,可以获得$N_G[S_L] = S_R$。

  $(2)\rightarrow (1)$: 假设$N_G[S_L] = S_R$ 并且 $\abs{S_L} = \abs{S_R}$。
  由于$N_G[S_L] = S_R$,那么$E_L$肯定是空集。
  因为$\abs{S_L} = \abs{S_R}$,那所有流到$S_W \cap R_W$的流量肯定都来自$S_W \cap L_W$,那么也可以得到$E_R$也为空集。故在网络W的残余图中$G_W(f)$中不存在从$S_W$到$(L_W \cup R_W) \setminus S_W$的边。

  证毕。
\end{proof}

观察上述引理\ref{residualgraph1}中,我们发现条件(2)与是“$S_L$是满足surplus值为0的独立集”的弱化条件,
接下来我们在两个条件中都增加约束,来获得更强的结论。
\begin{lemma} \label{residualgraph2}
如果网络W的流量残余图$G_W(f)$中,有一个尾强连通分量$S_W$满足$S_L \cap S_R = \emptyset$,
那么$S_L$是图G中的一个独立集并且$\abs{S_L} = \abs{N_G(S_L)}$。
\end{lemma}
\begin{proof}
$S_W$是$G_W(f)$中一个尾强连通分量等价于$G_W(f)$中不存在从$S_W$到$(L_W \cup R_W) \setminus S_W$的边。
故依据引理\ref{residualgraph1},可以获得$N_G[S_L] = S_R$ 并且 $\abs{S_L} = \abs{S_R}$。
又因为$S_L \cap S_R = \emptyset$,可以得知$N_G(S_L) = N_G[S_L] = S_R$ 并且 $S_L$是图G中一个独立集。证毕。
\end{proof}

\begin{lemma} \label{residualgraph3}
如果图G中存在独立集$T \subseteq V$满足$\abs{T} = \abs{N_G(T)}$,
那么网络W在最大可行流$f$下的残余图$G_W(f)$中存在一个尾强连通分量$S_W$满足$S_L \cap S_R = \emptyset$。
\end{lemma}
\begin{proof}
令$T$表示图G中满足$\abs{T} = \abs{N_G(T)}$的最小独立集。
构造$S_W = L_T \cup R_{N_G(T)}$,根据定义$S_L = T, S_R = N_G(T)$。
因为$T$是图G中满足$\abs{T} = \abs{N_G(T)}$的独立集,
我们可以获得$N_G(T) = N_G[T] = N_G[S_L] = S_R$。
可以进一步整理得引理\ref{residualgraph1}中的条件(2), $N_G[S_L] = S_R$ 并且 $\abs{S_L} = \abs{S_R}$。
故$G_W(f)$中不存在从$S_W$到$(L_W \cup R_W) \setminus S_W$的边。

假设$S_W$不是网络W在最大可行流$f$下的残余图$G_W(f)$中的强连通分量,
则存在$S_W' \subset S_W$使得$G_W(f)$中不存在从$S_W'$到$S_W \setminus S_W'$的边。
又因为之前已经发现$G_W(f)$中不存在从$S_W$到$(L_W \cup R_W) \setminus S_W$的边,
两者结合可以进一步发现$G_W(f)$中不存在从$S_W'$到$(L_W \cup R_W) \setminus S_W'$的边。
根据引理\ref{residualgraph2},可以发现$S_L'$是图G中的一个独立集并且$\abs{S_L'} = \abs{N_G(S_L')}$。
因为$S_W'$是$S_W$的真子集,故$S_L' \subset T$,这与T是满足$\abs{T} = \abs{N_G(T)}$的最小独立集矛盾。
故假设不成立,$S_W$是网络W的残余图$G_W(f)$中的强连通分量。

综上,$S_W$是网络W的残余图$G_W(f)$中的尾强连通分量,且$S_L \cap S_R = \emptyset$。

证毕。
\end{proof}

引理\ref{residualgraph2}和引理\ref{residualgraph3},为我们提供了俄寻找图G中满足$surplus$值为0的独立集的方法。
当我们从残余图$G_W(f)$中找到一个尾强连通分量$S_W$满足$S_L \cap S_R = \emptyset$,
则可以判断$S_L$是图G的独立集,且满足$surplus(S_L) = 0$。
通过将$N(S_L)$中顶点选进顶点覆盖集,可以获得新的VCAL问题实例如下,
\[(G' = G[V \setminus (S_L \cup N(S_L))], k' = k - \abs{N(S_L)})\]
对于求解图$G'$上Dual-LPVC问题的网络$W'$,可以简单通过从原本网络W中删除点集$S_W$以及相应的边获得。
现在剩下的唯一的问题就是,如何快速获得网络$W'$的最大可行流以及对应的新的残余图。
为了解决该问题,这里给出下面结论:
\begin{claim}
假设函数$f':E' \rightarrow \mathbb{R}$, 是通过将原网络W最大可行流$f$定义域限制为$E'$获得的,
即对任意$e \in E'$有$f'_e = f_e$。
\begin{itemize}
  \item {(1)} $f'$是网络$W'$的最大可行流;
  \item {(2)} 网络$W'$在流量$f'$下的残余图$G_W'(f')$等价于原残余图$G_W(f)$删除点集$S_W$以及相应的边。
\end{itemize}
\end{claim}
\begin{proof}

(1)
首先在网络$W'$上,对任意边e,由于$f'(e), c'(e)$均继承于网络W,显然有$0 \le f'(e) \le c'(e)$。

因为全$\frac{1}{2}$函数是图G的Primal-LPVC最优解,所以网络W的最大可行流流量$F = 2 * val(x) = \abs{V}$。
因此在网络W中,对于任何从源点s出发或者到达汇点t的边e有$f_e = 1$,显然该性质在$W', f'$中也成立。
故在网络$W'$上
\begin{equation*}
    \sum_{e\in \delta^+(s)}{f'(e)} =\sum_{e\in \delta^-(t)}{f'(e)} = \abs{V'}
\end{equation*}

\textcolor[rgb]{1.00,0.00,0.00}{补充图}

回顾引理\ref{residualgraph1}证明,
我们已经证明了当残余图$G_W(f)$中不存在$S_W$到$(L_W \cup R_W) \setminus S_W$ 的边时候,
$E_L = \{(u, v) \in E_W\;|\;u \in S_W,v \in (L_W \cup R_W) \setminus S_W\}=\emptyset$,
$E_R = \{(v, u)\;|\;(u, v) \in E_W,f(u, v) > 0,u \in (L_W \cup R_W) \setminus S_W,v \in S_W\}=\emptyset$。
所以从网络$W$上删除$S_W$以及相应边后,对于任何剩下的非$s, t$顶点,显然流入的流量以及流出的流量都没有改变。
故在网络$W'$上有,
\begin{equation*}
    \sum_{e\in \delta^+(v)}{f'(e)}=\sum_{e\in \delta^-(v)}{f'(e)}\mbox{, 对于任何} v \in V' \setminus \{s, t\}
\end{equation*}

结合以上三点可知$f'$是网络$W'$上的可行流,又因为不可能存在更大的流,所以$f'$是网络$W'$上的最大可行流。

(2)
由网络$W'$以及其最大可行流$f'$的定义,显然可以得知$G_W'(f')$等价于原残余图$G_W(f)$删除点集$S_W$以及相应的边。

证毕。
\end{proof}

\subsection{完整核心化算法}
\textcolor[rgb]{1.00,0.00,0.00}{need}

\section{分支策略}





\section{算法有效性证明}
\section{算法复杂度分析及小结}
\section{小结}