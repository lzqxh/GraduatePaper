%=========================================================================%
%   LaTeX File for PhD thesis of Dalian University of Technology
%-------------------------------------------------------------------------%
%   Copyright 2003  by     (FBI@BHQT)
%=========================================================================%

%=========================================================================%
%                          主文档 格式定义
%=========================================================================%

%===================== 重定义字体、字号命令 =============================%
% 注意win2000,没有 simsun, 最好到网上找一个。一些字体是office2000带的
\setCJKmainfont{SimSun}

\setCJKfamilyfont{hei}{SimHei}
\newcommand{\hei}{\CJKfamily{hei}}                          %黑体
\setCJKfamilyfont{kai}{KaiTi}
\newcommand{\kai}{\CJKfamily{kai}}                          %行楷
\setCJKfamilyfont{fang}{FangSong}
\newcommand{\fang}{\CJKfamily{fang}}                        %仿宋
\setCJKfamilyfont{song}{SimSun}
\newcommand{\song}{\CJKfamily{song}}                        %宋体 song
%\setCJKfamilyfont{xingkai}{FZXingKai-S04}
%\newcommand{\xingkai}{\CJKfamily{xingkai}}                  %华文行楷



\newcommand{\chuhao}{\fontsize{42pt}{\baselineskip}\selectfont}     % 字号设置
\newcommand{\xiaochuhao}{\fontsize{36pt}{\baselineskip}\selectfont} % 字号设置
\newcommand{\xiaoyi}{\fontsize{24pt}{\baselineskip}\selectfont}     % 小一, 单倍行距
\newcommand{\yihao}{\fontsize{28pt}{\baselineskip}\selectfont}      % 字号设置
\newcommand{\erhao}{\fontsize{21pt}{\baselineskip}\selectfont}      % 字号设置
\newcommand{\xiaoerhao}{\fontsize{18pt}{\baselineskip}\selectfont}  % 字号设置
\newcommand{\sanhao}{\fontsize{15.75pt}{\baselineskip}\selectfont}  % 字号设置
\newcommand{\xiaosanhao}{\fontsize{15pt}{\baselineskip}\selectfont} % 字号设置
\newcommand{\sihao}{\fontsize{14pt}{\baselineskip}\selectfont}      % 字号设置
\newcommand{\xiaosihao}{\fontsize{12pt}{\baselineskip}\selectfont}  % 字号设置
\newcommand{\wuhao}{\fontsize{10.5pt}{\baselineskip}\selectfont}    % 字号设置
\newcommand{\xiaowuhao}{\fontsize{9pt}{\baselineskip}\selectfont}   % 字号设置
\newcommand{\liuhao}{\fontsize{7.875pt}{\baselineskip}\selectfont}  % 字号设置
\newcommand{\qihao}{\fontsize{5.25pt}{\baselineskip}\selectfont}    % 字号设置

%===================================================================%
%                         各种距离与缩进
%===================================================================%

%-------------------- 用于中文段落缩进 和正文版式 ------------------%

\setlength{\parindent}{2em}                 % 首行两个汉字的缩进量
\setlength{\parskip}{3pt plus1pt minus1pt}  % 段落之间的竖直距离
\renewcommand{\baselinestretch}{1.25}\normalsize      % 定义行距

%------------------------- 列表与图表距离设置 -----------------------%
\setlength{\topsep}{3pt plus1pt minus2pt}           % 第一个item和前面版落间的距离
\setlength{\partopsep}{3pt plus1pt minus2pt}        % 当在一个新页开始时加到
                                                    % \topsep的额外空间
\setlength{\itemsep}{3pt plus1pt minus2pt}          % 连续items之间的距离.
\setlength{\floatsep}{10pt plus 3pt minus 2pt}      % 图形之间或图形与正文之间的距离
\setlength{\abovecaptionskip}{2pt plus1pt minus1pt} % 图形中的图与标题之间的距离
\setlength{\belowcaptionskip}{3pt plus1pt minus2pt} % 表格中的表与标题之间的距离

%下面这组命令使浮动对象的缺省值稍微宽松一点,从而防止幅度
%对象占据过多的文本页面,也可以防止在很大空白的浮动页上放置
%很小的图形。
\renewcommand{\textfraction}{0.15}
\renewcommand{\topfraction}{0.85}
\renewcommand{\bottomfraction}{0.65}
\renewcommand{\floatpagefraction}{0.60}

%---------------------------- 数学公式设置 ------------------------------%
\setlength{\abovedisplayskip}{5pt plus1pt minus1pt}     %公式前的距离
\setlength{\belowdisplayskip}{5pt plus1pt minus1pt}     %公式后面的距离
\setlength{\arraycolsep}{2pt}   %在一个array中列之间的空白长度, 因为原来的太宽了

\allowdisplaybreaks[4]  % \eqnarray如果很长,影响分栏、换行和分页
                        %(整块挪动,造成页面空白),可以设置成为自动调整模式

\renewcommand{\mathbf}[1]{\boldsymbol{#1}}  %重新定义 \mathbold 为矢量黑斜体
                                            %用Mathtype生成的公式源码直接应用于程序中
                                            %(导出时请选择 TeX-AMS-LaTeX )

\newcommand{\me}{\mathrm{e}}  %定义 对数常数e,虚数符号i,j以及微分算子d为直立体。
\newcommand{\mi}{\mathrm{i}}
\newcommand{\mj}{\mathrm{j}}
\newcommand{\dif}{\mathrm{d}}

%===================================================================%
%                         各种标题样式
%===================================================================%
%======================= 标题名称中文化 ============================%
\renewcommand\contentsname{目\ 录}
\renewcommand\listfigurename{插\ 图\ 目\ 录}
\renewcommand\listtablename{表\ 格\ 目\ 录}
%\renewcommand\abstractname{摘\ 要} %err undefined
%\renewcommand\refname{参考文献}         %article类型
%\renewcommand\bibname{\song\sanhao{\textbf{参\ 考\ 文\ 献}}}    %book类型
\renewcommand\indexname{索\ 引}
\renewcommand\figurename{图}
\renewcommand\tablename{表}
\renewcommand\partname{第\CJKnumber{\value{part}}部分}
\renewcommand\chaptername{第\CJKnumber{\value{chapter}}章}
%\renewcommand\chaptername{\CJKprechaptername\CJKthechapter\CJKchaptername}
%======================= 定制章节的标题样式 =============================%
\setcounter{secnumdepth}{3}
%---------------------- 定义章节的编号格式 --------------------------%
%\renewcommand{\thesection}{\CJKnumber{\arabic{section}}、} % 定义 一、。九、十
\renewcommand{\thesection}{\arabic{chapter}.\arabic{section}}
\renewcommand{\thesubsection}{\arabic{chapter}.\arabic{section}.\arabic{subsection}}
\renewcommand{\thesubsubsection}{\arabic{subsubsection}.}
\let\oldtitle=\title
\def\title#1{\oldtitle{\cnbf{#1}}}

%\CJKtilde   %用于解决英文字母和汉字的间距问题。例如:变量~$x$~的值。
            %这个命令重新定义~。等价于 \def~{\hspace{0.25em plus 0.125em minus 0.08em}}
            %若想恢复~,则用命令 \standardtilde 。

\renewcommand{\CJKglue}{\hskip 0pt plus 0.08\baselineskip}
%\CJKglue   %这个命令原始定义为
            %newcommand{\CJKglue}{\hskip 0pt plus 0.08\baselineskip}
            %它于必要时在汉字之间插入一个附加的空隙,以解决行的超长问题。
            %可以修改此命令,增加它的值,以加强其调节能力。
            %注意,使用这个命令可能导致出现空白页!!!

%----------------------- 定义章节标题格式 ----------------------------%

%\titleformat{\chapter}[hang]{\xiaoerhao\filcenter\CJKfamily{hei}\Large}
%        {\Large \chaptertitlename}{10em}{}{}
\titleformat{\chapter}[hang]{\normalfont\sanhao\filcenter\CJKfamily{hei}}
    {\sanhao{\chaptertitlename}}{20pt}{\huge}
\titlespacing{\chapter}{0pt}{-3ex  plus .1ex minus .2ex}{2.5ex plus .1ex minus .2ex}

%\titleformat{\section}[hang]{\CJKfamily{hei}\Large \centering} %标题居中
\titleformat{\section}[hang]{\sanhao\CJKfamily{hei}}
    {\sihao\thesection}{1em}{}{}
\titlespacing{\section}
    {0pt}{1.5ex plus .1ex minus .2ex}{\wordsep}

\titleformat{\subsection}[hang]{\sihao\CJKfamily{hei}}
    {\sihao\thesubsection}{1em}{}{}
\titlespacing{\subsection}%
    {0pt}{1.5ex plus .1ex minus .2ex}{\wordsep}

\titleformat{\subsubsection}[hang]{\CJKfamily{hei}}
    {\thesubsubsection }{1em}{}{}
\titlespacing{\subsubsection}%
    {0pt}{1.2ex plus .1ex minus .2ex}{\wordsep}

%======================= 定义列表项目格式 ==========================%
\renewcommand\labelenumi{\textcircled{\scriptsize \theenumi}}  %带圈的数字
%\renewcommand\labelenumi{(\theenumi)}
\renewcommand\labelenumii{(\theenumii)}
\renewcommand\labelenumiii{\theenumiii.}
\renewcommand\labelenumiv{\theenumiv.}

%====================== 定制图形和表格标题样式 =====================%
%---------------------- 定制图形和表格标题格式 ---------------------%
\renewcommand{\captiondelim}{\ } %定义如  "图(表)2: 示例" 中的间隔符号,如 ":" ,这里定义为空
%\renewcommand{\captionlabelsep}{\hspace{1em}} %定义图表编号与标题间的间隔距离
%\renewcommand{\captionlabelfont}{\small \CJKfamily{hei}\bf} %定义图表标签的字体
%\renewcommand{\captionfont}{\small \CJKfamily{song}\rmfamily} %定义图表标题内容的字体
%% \scriptsize \footnotesize \small \large \Large  %图形标签字体大小


%--------------------- 定义图、表、公式的编号格式 -------------------%
\renewcommand{\thetable}{\arabic{chapter}-\arabic{table}}
%\renewcommand{\theequation}{\arabic{chapter}-\arabic{equation}}
\renewcommand{\thefigure}{\arabic{chapter}-\arabic{figure}}


%=========================== 目录设置 ==================================%
\setcounter{tocdepth}{2} \setcounter{secnumdepth}{2}
% 可以用\section[abc]{abcdefg}形式的命令,这样abc就做为缩短标题出现
% 在目录表中和页眉上.另外,还可以利用\addtocounter{secnumdepth}{num}
% 来使得当前章节编号深度增加或减小,num可取正值或负值.
%*******取自ctex           定义目录格式   启用titletoc宏包
%%******序号,标题间距等
\titlecontents{chapter}[2.em]{\color{black}\normalsize}%\CJKfamily{hei}\addvspace{4ex}}
 {\contentslabel{2em}\hspace*{1.0em}}{\hspace*{-2.3em}}
 {\color{black}\titlerule*[0.4pc]{.}\contentspage}[\addvspace{0.5ex}]

\titlecontents{section}
[3.4em] {\color{black}\normalsize}
{\contentslabel{2em}\hspace*{-0.0em}} {\hspace*{-2.3em}}
{\color{black}\titlerule*[0.4pc]{.}\contentspage}

\titlecontents{subsection}
[5.4em] {\color{black}\normalsize}
{\contentslabel{2em}\hspace*{1.0em}} {\hspace*{-3.3em}}
{\color{black}\titlerule*[0.4pc]{.}\contentspage}
%%%%%%%%%%%%%%%%%%%%%%%%%%%%%%%%%%%%%%%%%%%%%%%%%%%%%%

%  缩小目录中各级标题之间的缩进
%\dottedcontents{chapter}[1.0em]{\hei\vspace{0.1em}}{1.0em}{5pt}
%\dottedcontents{section}[2.2cm]{}{2.5em}{5pt}
%\dottedcontents{subsection}[3.0cm]{}{2.5em}{5pt}
%\dottedcontents{subsubsection}[2.86cm]{}{3.4em}{5pt}


%============================= 页面设置 ================================%
%-------------------- 定义页眉和页脚 使用fancyhdr 宏包 -----------------%
% 定义页眉与正文间双隔线
\iffalse
\newcommand{\makeheadrule}{%
    \makebox[-3pt][l]{\rule[.7\baselineskip]{\headwidth}{0.4pt}}
    \rule[0.85\baselineskip]{\headwidth}{1.5pt}\vskip-.8\baselineskip}
\makeatletter
\renewcommand{\headrule}{%
    {\if@fancyplain\let\headrulewidth\plainheadrulewidth\fi
     \makeheadrule}}
\makeatother
\fi

%**************重定义在页眉处画一单横线
\newcommand{\makeheadrule}{%
    \makebox[-3pt][l]{\rule[0.7\baselineskip]{\headwidth}{0.4pt}}}
%    \rule[0.85\baselineskip]{\headwidth}{1.5pt}\vskip-.8\baselineskip}
\makeatletter
\renewcommand{\headrule}{%
    {\if@fancyplain\let\headrulewidth\plainheadrulewidth\fi
     \makeheadrule}}
\makeatother

%如果需要画单隔线,则需要
\iffalse%-------------------------------%
\renewcommand{\headrulewidth}{0.5pt}    %在页眉下画一个0.5pt宽的分隔线
\renewcommand{\footrulewidth}{0pt}      % 在页脚不画分隔线。
\fi%-------------------------------------%

\pagestyle{fancyplain}

\renewcommand{\chaptermark}[1]%
{\markboth{\chaptername \ #1}{}}            % \chaptermark 去掉章节标题中的数字
\renewcommand{\sectionmark}[1]%
{\markright{\thesection \ #1}{}}            % \sectionmark 去掉章节标题中的数字
\fancyhf{}  %清除以前对页眉页脚的设置

\fancyhead[CO]{\color{black}\CJKfamily{song}\wuhao\ctitle\hfill\leftmark}     % 在book文件类别下,
\fancyhead[CE]{\color{black}\CJKfamily{song}\wuhao\ctitle\hfill\leftmark}     % \leftmark自动存录各章之章名,
%\fancyhead[C]{\color{blue}\CJKfamily{fs}{\mytitle}}       %此处替换论文的题目
%\fancyfoot[L]{\CJKfamily{fs}hdwyn@sohu.com}             % \rightmark记录节标题
\fancyfoot[C]%                                          % [RE][LO]
{\CJKfamily{hei} -~\thepage~-}
%{\CJKfamily{hei} 第\thepage页,共\pageref{LastPage}页}
%\fancyfoot[R]{\CJKfamily{fs}学位论文模板}

%如果要要改变封面和章首页的页眉和页脚,则需要
\iffalse%----------------------------------------%
\fancypagestyle{plain} {
\fancyhead{}                                    % clear all header fields
\fancyhead[CE,CO]{这是章首页或封面的页眉}
\renewcommand{\headrulewidth}{0pt}
\fancyfoot{}                                    % clear all footer fields
\fancyfoot[CE,CO]{\thepage} }
\fi%--------------------------------------------%

%=== 配合前面的ntheorem宏包产生各种定理结构,重定义一些正文相关标题 ===%
\theoremstyle{plain}
\theoremheaderfont{\normalfont\rmfamily\CJKfamily{hei}}
%\theorembodyfont{\normalfont\rm\CJKfamily{song}} \theoremindent0em
\theorembodyfont{\normalfont\rm\CJKfamily{kai}} \theoremindent0em
\theoremseparator{\hspace{1em}} \theoremnumbering{arabic}
%\theoremsymbol{}          %定理结束时自动添加的标志
\newtheorem{definition}{\hspace{2em}定义}[chapter]
%\newtheorem{definition}{\hei 定义}[section] %!!!注意当section为中国数字时,[sction]不可用!
\newtheorem{proposition}{\hspace{2em}命题}[chapter]
\newtheorem{property}{\hspace{2em}性质}[chapter]
\newtheorem{lemma}{\hspace{2em}引理}[chapter]
%\newtheorem{lemma}[definition]{引理}
\newtheorem{theorem}{\hspace{2em}定理}[chapter]
\newtheorem{axiom}{\hspace{2em}公理}[chapter]
\newtheorem{corollary}{\hspace{2em}推论}[chapter]
\newtheorem{claim}{\hspace{2em}断言}[chapter]
\newtheorem{exercise}{\hspace{2em}习题}[chapter]

\newtheorem{reducerule}{\hspace{2em}收缩规则}
\theoremsymbol{$\blacksquare$}
\newtheorem{example}{\hspace{2em}例}[chapter]

\theoremstyle{nonumberplain}
\theoremheaderfont{\CJKfamily{hei}\rmfamily}
\theorembodyfont{\normalfont \rm \CJKfamily{song}}
\theoremindent0em \theoremseparator{\hspace{1em}}
\theoremsymbol{$\blacksquare$}
\newtheorem{proof}{\hspace{2em}证明}
\newtheorem{branchrule}{\hspace{2em}分支规则}

%=========================== 修改引用的格式 ==============================%
% 第一行在引用处数字两边加方框
% 第二行去除参考文献里数字两边的方框
%\makeatletter
%\def\@cite#1{\mbox{$\m@th^{\hbox{\@ove@rcfont[#1]}}$}}
%\renewcommand\@biblabel[1]{#1}
%\makeatother
% 增加 \upcite 命令使显示的引用为上标形式
%\newcommand{\upcite}[1]{$^{\mbox{\scriptsize \cite{#1}}}$}             % 方法1
\newcommand{\upcite}[1]{\textsuperscript{\textsuperscript{\cite{#1}}}}  % 方法2
%\makeatletter
%\def\@cite#1#2{\textsuperscript{[{#1\if@tempswa,#2\fi}]}}
%\makeatother
%=============================== 脚注 =============================%
\renewcommand{\thefootnote}{\arabic{footnote}}
%detcounter{footnote}{0}

%==================== 定义题头格言的格式 ==========================%
% 用法 \begin{Aphorism}{author}
%         aphorism
%      \end{Aphorism}
\newsavebox{\AphorismAuthor}
\newenvironment{Aphorism}[1]
{\vspace{0.5cm}\begin{sloppypar} \slshape
\sbox{\AphorismAuthor}{#1}
\begin{quote}\small\itshape }
{\\ \hspace*{\fill}------\hspace{0.2cm} \usebox{\AphorismAuthor}
\end{quote}
\end{sloppypar}\vspace{0.5cm}}

%============================== 控制表格线宽 ==========================%
% 更改横线(\hline)线宽:定义如下命令\hlinewd代替\hline。
% 更改垂直线(\vline)线宽:使用\usapackage{array},则可以在指定垂直线的地方用
% “!{\vrule width 3.5pt}”代替“|”,如“|c!{\vrule width 5pt}p{5cm}|r|”

\makeatletter
\def\hlinewd#1{%
  \noalign{\ifnum0=`}\fi\hrule \@height #1 \futurelet
   \reserved@a\@xhline}
\makeatother
\newcommand\vlinewd[1][1pt]{\vrule width #1}

% 不过上面的命令\hlinewd不能与longtable正常工作(reported by %钟圣俊老师),
% 只能使用下面的方法实现线宽控制:
%
%\setlength{\arrayrulewidth}{0.5pt}
%\setlength{\doublerulesep}{\arrayrulewidth}
%\newcommand{\dhline}{\hline\hline}
%\newcommand{\thline}{\hline\hline\hline}
%(类似的可以定义更多不同宽度的\hline)


%========================== 其它自定义 ==============================%
%====================================================================%
% 下面定义的命令(\alpheqn \reseteqn)可以使公式编号变为 4-a,4-b
% 使用说明:\alpheqn 为开始产生处,\reseteqn为恢复原来公式编号形式处
% 这两个命令为自定义,使用时应注意:不可放于 数学环境中!!!
% 在公式开始前和结束后使用!!!
%====================================================================%
%\newcounter{saveeqn}%
%
%\newcommand{\alpheqn}{%
%\setcounter{saveeqn}{\value{equation}}%
%\stepcounter{saveeqn}%
%\setcounter{equation}{0}%
%%\renewcommand{\theequation}{\arabic{saveeqn}-\alph{equation}}}%%article 中的定义
%%\renewcommand{\theequation}{\arabic{chapter}-\arabic{saveeqn}\alph{equation}}}%book %中的定义
%%{\mbox{\arabic{equation}-\alph{equation}}}}%
%
%\newcommand{\reseteqn}{%
%\setcounter{equation}{\arabic{chapter}-\value{saveeqn}}%
%%%\renewcommand{\theequation}{\arabic{equation}}}    %article 中的定义
%\renewcommand{\theequation}{\arabic{chapter}-\arabic{equation}}}  %book 中的定义
\numberwithin{equation}{chapter}
%====================================================================%
% 下面定义的命令(\alphfig \resetfig)可以使插图编号变为 4-a,4-b
% 使用说明:\alphfig 为开始产生处,\resetfig为恢复原来插图编号形式处
% 这两个命令为自定义,使用时应注意:不可放于 数学环境中!!!
% 在插图开始前和结束后使用!!!
%====================================================================%
\newcounter{savefig}%

\newcommand{\alphfig}{%
\setcounter{savefig}{\value{figure}}%
\stepcounter{savefig}%
\setcounter{figure}{0}%
%%\renewcommand{\thefigure}{\arabic{savefig}-\alph{figure}}}%%article 中的定义
\renewcommand{\thefigure}{\arabic{chapter}-\arabic{savefig}\alph{figure}}}%book 中的定义
%{\mbox{\arabic{figure}-\alph{figure}}}}%

\newcommand{\resetfig}{%
\setcounter{figure}{\value{savefig}}%
%%\renewcommand{\thefigure}{\arabic{figure}}}    %article 中的定义
\renewcommand{\thefigure}{\arabic{chapter}-\arabic{figure}}}  %book 中的定义

%====================================================================%
% 下面定义的命令(\alphtab \resettab)可以使表格编号变为 4-a,4-b
% 使用说明:\alphtab 为开始产生处,\resettab为恢复原来表格编号形式处
% 这两个命令为自定义,使用时应注意:不可放于 数学环境中!!!
% 在表格开始前和结束后使用!!!
%====================================================================%
\newcounter{savetab}%

\newcommand{\alphtab}{%
\setcounter{savetab}{\value{table}}%
\stepcounter{savetab}%
\setcounter{table}{0}%
%%\renewcommand{\thetable}{\arabic{savetab}-\alph{table}}}%%article 中的定义
\renewcommand{\thetable}{\arabic{chapter}-\arabic{savetab}\alph{table}}}%%book 中的定义
%{\mbox{\arabic{table}-\alph{table}}}}%

\newcommand{\resettab}{%
\setcounter{table}{\value{savetab}}%
%%\renewcommand{\thetable}{\arabic{table}}}    %article 中的定义
\renewcommand{\thetable}{\arabic{chapter}-\arabic{table}}}  %book 中的定义

%====================================================================%
% 自定义项目列表标签及格式 \begin{mylist} 列表项 \end{mylist}
%====================================================================%
\newcounter{newlist} %自定义新计数器
\newenvironment{mylist}[1][可改变的列表题目]{%%%%%定义新环境
\begin{list}{\textbf{\hei #1} \arabic{newlist}:} %%标签格式
    {
     \usecounter{newlist}
     \setlength{\labelwidth}{22pt} %标签盒子宽度
     \setlength{\labelsep}{0cm} %标签与列表文本距离
     \setlength{\leftmargin}{0cm} %左右边界
     \setlength{\rightmargin}{0cm}
     \setlength{\parsep}{0.5ex plus0.2ex minus0.1ex} %段落间距
     \setlength{\itemsep}{0ex plus0.2ex} %标签间距
     \setlength{\itemindent}{44pt} %标签缩进量
     \setlength{\listparindent}{22pt} %段落缩进量
    }}
{\end{list}}%%%%%

%% 自己重定义一个双标题的命令
%% 使用: \figcaption{中文标题}{English titile}
\newcommand{\figcaption}[2]{
    \renewcommand\figurename{图}
    \caption{#1}
    \addtocounter{figure}{-1}
    \renewcommand\figurename{Fig.}
    \caption{#2}
}
\newcommand{\tablecaption}[2]{
    \renewcommand\tablename{表}
    \caption{#1}
    \addtocounter{table}{-1}
    \renewcommand\tablename{Tab.}
    \caption{#2}
}


\newcommand{\abs}[1]{\left\vert #1 \right\vert}

%% 罗马数字
\makeatletter
\newcommand{\rmnum}[1]{\romannumeral #1}
\newcommand{\Rmnum}[1]{\expandafter\@slowromancap\romannumeral #1@}
\makeatother



