
%%%%%%%%%%%%%%%%%%%%%%%%%%%%%%%%%%%%%%%%%%%%%%%%%%%%%%%%%%%%%%%%%%%%%%%%%%%%%
%   Copyright 2004  by     (FBI@BHQT  )
%%%%%%%%%%%%%%%%%%%%%%%%%%%%%%%%%%%%%%%%%%%%%%%%%%%%%%%%%%%%%%%%%%%%%%%%%%%%%
%\thispagestyle{empty}
%\thispagestyle{plain}
%\addcontentsline{toc}{chapter}{\hspace*{0.2cm}摘~ 要}
\phantomsection\addcontentsline{toc}{chapter}{\hspace*{0.2cm}摘~要}
\setlength{\headheight}{26pt}

\chapter*{ \markboth{摘要}{摘要}}
%\chapter{绪论}
\setcounter{page}{1}
\renewcommand{\thepage}{\Roman{page}}
\vspace*{-1.5cm}

论文题目: \,\ctitle

专\hspace{2em}业: \,\major

硕 \, 士  \, 生: \,\cauthor

指导教师: \,\supervisor

\vspace{1.5cm}
 \centerline{\xiaoerhao{\hei{摘\qquad  要}}}
\vspace{0.5cm}

因特网流量分类研究是众多因特网研究的基础,清楚地了解整个因特网的流量情况对于因特网流量建模、
网络运行维护管理、
网络安全及流量工程等均具有重要意义。在~P2P~应用逐渐普及的今天,
基于端口和载荷的传统流量分类方法逐渐凸现出局限性,
迫切需要一种新型的、 有效的流量分类方法, 为因特网业务的~QoS~保证、
网络异常检测等提供支撑。


\vspace{1cm}

{\noindent\hei \textbf{关键词}}: {因特网流量分类}
