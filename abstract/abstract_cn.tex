
%%%%%%%%%%%%%%%%%%%%%%%%%%%%%%%%%%%%%%%%%%%%%%%%%%%%%%%%%%%%%%%%%%%%%%%%%%%%%
%   Copyright 2004  by     (FBI@BHQT  )
%%%%%%%%%%%%%%%%%%%%%%%%%%%%%%%%%%%%%%%%%%%%%%%%%%%%%%%%%%%%%%%%%%%%%%%%%%%%%
%\thispagestyle{empty}
%\thispagestyle{plain}
%\addcontentsline{toc}{chapter}{\hspace*{0.2cm}摘~ 要}
\phantomsection\addcontentsline{toc}{chapter}{\hspace*{0.2cm}摘~要}
%\setlength{\headheight}{26pt}

%\chapter*{ \markboth{摘要}{摘要}}
%\chapter{绪论}
\setcounter{page}{1}
\renewcommand{\thepage}{\Roman{page}}
%\vspace*{-1.5cm}
\setlength{\parindent}{0em}
论文题目: \,\ctitle

专\hspace{2em}业: \,\major

硕\hspace{0.5em}士\hspace{0.5em}生: \,\cauthor

指导教师: \,\supervisor
\setlength{\parindent}{2em}

\vspace{1.2cm}
 \centerline{\xiaoerhao{\hei{摘\qquad  要}}}
\vspace{0.3cm}
对于NP难问题,如果能基于参数$k$及问题规模$n$设计出时间复杂度为$O(f(k)n^{O(1)})$的算法,则称该问题是固定参数可解的。
在理论计算机科学中,为NP-难问题设计实际有效的固定参数算法是当前研究中一个重要的方向。
本文以顶点覆盖、反馈顶点集两个经典的NP-难问题为研究对象,通过挖掘问题的性质,应用多种参数计算技术分别为之设计了相应的固定参数算法。
值得一提的是,本文中所有研究都是基于一般无向图,更加具有普遍性。

在顶点覆盖问题上,我们设计了基于线性松弛下界的参数化顶点覆盖问题的固定参数算法。
该问题以目标顶点覆盖集大小和最小顶点覆盖集的线性松弛下界的差值作为算法的参数$\mu$。
我们首先使用最大网络流来维护图的顶点覆盖集的线性松弛下界,
并且在最大流的残余图上线性地将问题实例核心化,
最后使用分支搜索树顺利获得一个时间复杂度为$O(2.618^\mu(\abs{V} + \abs{E}))$的线性固定参数算法。
对比之前该方向最好的研究成果,Iwata等\upcite{iwata2014linear}提出的时间复杂度为$O(4^\mu(\abs{V} + \abs{E}))$的算法,复杂度上有了明显地下降。

在反馈顶点集问题上,我们设计了一个复杂度为$\mathcal{O}^*(3.598^k)$的固定参数算法。
其主要思路是,首先使用迭代压缩技术将问题转化成不相交反馈顶点集问题,
之后我们使用分支搜索树来求解不相交反馈顶点集问题。在其中我们创新性地同时使用两个参数限制搜索树的规模。
当前在反馈顶点集问题上最好的固定参数算法是Kociumaka等\upcite{2013arXiv1306.3566K}提出的,复杂度是$\mathcal{O}^*(3.592^k)$。
与之对比,时间复杂度上十分接近,然而相比之下Kociumaka的算法一共使用了$12$条分支规则,而我们的算法只有$1$条更加简洁明了。

\vspace{1cm}

{\noindent\hei \textbf{关键词}}: {参数计算,固定参数算法,NP难问题, 顶点覆盖,反馈顶点集}
