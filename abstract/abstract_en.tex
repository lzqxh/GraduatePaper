
%%%%%%%%%%%%%%%%%%%%%%%%%%%%%%%%%%%%%%%%%%%%%%%%%%%%%%%%%%%%%%%%%%%%%%%%%%%%%
%   Copyright 2004  by      (FBI@BHQT)
%%%%%%%%%%%%%%%%%%%%%%%%%%%%%%%%%%%%%%%%%%%%%%%%%%%%%%%%%%%%%%%%%%%%%%%%%%%%%
\phantomsection\addcontentsline{toc}{chapter}{Abstract}

%\chapter*{ \markboth{Abstract}{Abstract}}

\vspace{-2.5cm}

\setlength{\parindent}{0em}
\begin{tabular}{ll}
  Title: & \etitle \\
  Major: & \emajor \\
  Name: & \eauthor \\
  Supervisor: & \esupervisor \\
\end{tabular}
\setlength{\parindent}{2em}

\vspace{1.5cm}
\begin{center}
{\hei\xiaoerhao \textbf{Abstract }}
\end{center}

对于NP-hard 问题,如果能基于参数k及问题规模n设计出时间复杂度为$O(f(k)n^{O(1)})$的算法,则称该问题是固定参数可解的。
在理论计算机科学中,为NP-hard问题设计实际有效的固定参数算法是当前研究中一个重要的方向。
本文以顶点覆盖、反馈顶点集两个经典的NP-hard问题为研究对象,通过挖掘问题的性质,应用多种参数计算技术分别为设计了相应的固定参数算法。
值得一提的是,本文中所有研究都是基于一般普通无向图,更加具有普遍性。

在顶点覆盖问题上,我们设计了基于线性松弛下界的参数化顶点覆盖问题的固定参数算法。
该问题以目标顶点覆盖集大小和最小顶点覆盖集的线性松弛下界的差值作为算法的参数$\mu$。
我们首先使用最大网络流来维护图的顶点覆盖集的线性松弛下界,
并且在最大流的残余图上线性地将问题实例核心化,
最后使用分支搜索树顺利获得一个时间复杂度为$O(2.618^\mu(\abs{V} + \abs{E}))$的线性固定参数算法。
对比之前Iwata等\upcite{iwata2014linear}在该问题上最好的时间复杂度$O(4^\mu(\abs{V} + \abs{E}))$有了明显提升。

在反馈顶点集问题上,我们设计了一个复杂度为$\mathcal{O}(3.598^k)$的固定参数算法。
其整理思路是,首先使用迭代压缩技术将问题转化成该问题的一个变种,不连通反馈顶点集问题,
之后我们同样使用分支搜索树来求解新问题。其中,我们创新性地同时使用两个参数限制搜索树的大小。
当前在反馈顶点集问题上最好的固定参数算法是Kociumaka\upcite{2013arXiv1306.3566K}提出的,复杂度是$\mathcal{O}(3.592^k)$。
与之对比,在时间复杂度上十分接近,然而相比之下Kociumaka的算法使用了12条分支规则,而我们的算法只有一条更加简洁明了。



\vspace{1cm} \noindent\textbf{Key Words}: Network traffic
classification;
