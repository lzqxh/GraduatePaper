\phantomsection

\addcontentsline{toc}{chapter}{\hspace*{0.2cm}致~谢}

\chapter*{ \markboth{致谢}{致谢}}
\vspace{-1cm}\centerline{\xiaoerhao{\hei{致\quad 谢}}}

\vspace{5mm}{\kai\xiaosihao
光阴飞逝,我在中山大学的第七年就这样逝去了。
这七年的学习和生活,让我从一个懵懂的高中毕业生成长到今天即将踏入社会工作的硕士。
这其中对母校的感恩,无以言表。
衷心感谢母校给我带来诸多优秀的师长,在学习上赋予我知识,
也衷心感谢母校让我认识了许多好友,在生活上予我以帮助关心。
可能这七年会是我生命中最重要的一段光阴!


在此时论文完成之际,我最想感谢的人是林瀚老师。
从大一在辩论队认识了林瀚师兄,到进入acm队遇到林瀚老师作为指导老师,
再到研究生阶段选择林瀚老师作为我的导师。
这期间林老师一直在我整个大学、研究生生涯中扮演着一个亦师亦兄的角色。
在各个方面,他给我的帮助实在太多了,难以一一列举,只能在此再说一声,谢谢!


另外一个需要感谢的人是我女朋友。在中大7年,与陈结贞就在一起了6年多。
我要感谢她这期间一直在背后给予我支持,支撑我完成自己的学业。

这里我还要感谢我的家人,爸爸、妈妈、爷爷、奶奶还有姐姐,
感谢他们养育了我这么多年,一直对我的学习予以无条件的支持。

最后我要感谢那些在这里不能一一提到的师长、同学和好友,感谢所有帮助和关心过我的人们。谢谢!

此刻真希望自己能有飞扬的文采将内心的感激之情更加写实的记录下来。
%}

\vspace{2cm}

\hfill {\kai 林泽群 \hspace{15mm}

\hfill 2016~年~5~月于中山大学}\hspace{5mm}
